

\begin{frame}{Results}
    \begin{itemize}
        \item We have studied kernel methods for electricity forecasting both in the point and probabilistic framework.
        \item We provided a Python open source implementation for kernel quantile regression compatible with the sklearn API. The code has been packaged and uploaded to the Python Package Index (PyPI) with the name \href{https://pypi.org/project/kernel-quantile-regression/\#2}{kernel-quantile-regression}. The \href{https://github.com/luca-pernigo/kernel_quantile_regression}{github repo} hosting the source code includes also the script implementing the experiments along with the cleaned datasets; this contribution is intended to forster reproducibility in research.
        \item We achieved superior performance of kernel quantile regression compared to standard quantile regressor algorithms. 
        \item We created and made available datasets suitable for algorithms benchmarking considering data from the DACH region (Germany, Switzerland and Austria). The format of these data takes inspiration from the popular GEFCom competitions.
        \item Kernel quantile regression extensive comparison between kernel function types.
        \item We compared kernel quantile regression against state of the art probabilistic algorithms in the literature by means of the GEFCom2014 competition.
    \end{itemize}
\end{frame}