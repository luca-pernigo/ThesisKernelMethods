Individuals and organizations constantly face situations of uncertainty, thus the need for robust forecasting methods. Such methods are crucial in the process of taking informed decisions and for strategic planning.
\\
The basic idea of forecasting is that we can extract knowledge from the past in order to make educated guesses about the future. Consequently, the range of fields where forecasting can be applied is very wide.
In this thesis, our focus lies on applying forecasting techniques to the energy sector. 
\\
Our decision to focus on the energy market is mainly motivated by the rapid changes it has undergone. %/experienced 
Over the last decades, electricity markets have gone through an unprecedented transformation. This shift was driven by the liberalization of such markets, the development and integration of renewable energy sources, the  increase of low carbon technologies and the adoption of smart meters. Events like the California electricity crisis are further motivating the choice of the electricity sector as subject of our studies, see \cite{california}.
Moreover, the process of deregulation lead to an increasing interest in the field of electricity forecasting (EF) within the academic community, see figure \ref{fig:epf_evolution}.
In addtion, the United Nations have identified the right to access affordable, reliable, sustainable and modern energy as one of their 17 sustainable development goals (SDGs) \cite{un_sdgs}.
Finally, the electricity market has a set of features that make it unique: electricity cannot be stored in an efficient way and supply and demand have to be matched instantly.
\\
\section{Motivation}
There are mutliple reasons why the energy sector needs robust forecasting techniques.
For power market companies, being able to predict prices with a low mean absolute percentage error (MAPE) \ref{mape} results in increased savings \cite{savings}. Furthermore, the adoption of smart meters provides power market companies with a huge amount of consumer data. This enables them to better model consumer preferences.
\\
Transmission system operators' (TSO) main goal is to match supply and demand, generally TSO do so by increasing or decreasing the generation. Thus, from their point of view forecasting is critical for balancing the electricity network.
Probabilistic forecasting may be useful to power producers, traders and consumers in order to improve their decision making process and managing risk. This holds in particular for traders, because probabilistic forecasts enable them to simulate scenarios and carry out stress tests.
Other possible applications include: control of storage, demand side response, anomaly detection, network design and planning, simulating inputs and handling missing data.
\\
\section{Point versus probabilistic forecast}
A distinction has to be made between two types of forecasting approaches: point forecasts and probabilistic forecasts.
Point prediction, also called deterministic forecasting in the literature \cite{EPF_review}, is all about predicting a particular value in time.
On the other hand, with probabilistic forecasting we aim at predicting either an interval, quantiles or a probability distribution for each point in time \cite{nowotarski}. For this reason, probabilistic forecasts are more informative than point forecasts. This is why the interest of the research community is shifting towards them.
A probabilistic forecast can be turned into a point forecast by simply taking its expectation.
Alternatively, a probabilistic forecast can be derived from a point one by modeling the residuals of the point prediction.



\section{Aims and objectives}

% - what is the question? objectives of the thesis
% the description of the problem tackled and the methodology used to solve
% it.

The scope of the thesis is analyzing state of the art forecasting methods in the energy market and to compare and to integrate them with ideas coming from the theory of kernel methods.  



\section{Outline}
We start with a literature review and a bibliometric analysis in Section \ref{literature_review}.
Then, the theory underlying kernel methods is covered in Section \ref{kernel_theory}. Evaluation metrics necessary to rank the forecasting techniques are presented in Section \ref{metrics}. Section \ref{energy_market} explains the core features and terminology of the energy market and of the electricity newtork.
Following, Sections \ref{ch:point} and \ref{ch:prob} introduce the state of the art methods in the context of point and probabilistic forecasting respectively.
Section \ref{eda} goes on with the extract-load-transform (ETL) pipeline and the exploratory data analysis (EDA). 
Implementation details are included in Section \ref{implementation}.
Finally, Section \ref{analysis} presents experiments results and discusses models' strenghts, weaknesses and possible improvements.