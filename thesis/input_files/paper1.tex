% Some commands used in this file

\section{\href{https://arxiv.org/abs/1605.09522}{Kernel Mean Embedding of Distributions: A Review and Beyond}}
From this first paper \cite{Muandet_2017}, the notation and terms used in the theory of Reproducing Kernel Hilbert Spaces are summarized.
\\
Many algorithms use the inner product as similarity measure between data instances $x, x' \in \mathcal{X}$. However, this inner product spans only the class of linear similarity measures. 
\\
The idea behind kernel methods is to apply a non-linear transformation $\phi$ to the data $x$ 
in order to get a more powerful non linear similarity measure.
\begin{align*}
\phi(x):\mathcal{X} &\longrightarrow \mathcal{F}
    \\
    x&\mapsto \phi(x)
\end{align*}


Then we take the inner product in the high dimensional space $\mathcal{F}$ mapped by $\phi(x)$.
\\
\begin{align*}
k(x,x'):=&\langle \phi(x), \phi(x') \rangle_{\mathcal{F}}
\end{align*}
\\
$\phi(x)$ is referred as feature map while $k$ as kernel function.
\\
Therefore, we can kernelize any algorithm involving a dot product by substituting $\langle x, x' \rangle_{\mathcal{X}}$ with $\langle \phi(x), \phi(x') \rangle_{\mathcal{F}}$
\\
One would expect constructing the feature maps explicitly and then evaluate their inner product in $\mathcal{F}$ to be computationally expensive, and indeed it is. However, we do not have to explicitly perform such calculations. This is because of the existence of the kernel trick.
To illustrate the idea behind the kernel trick consider the following example. 
\\
Suppose $x \in \mathbb{R}^2$ and assume to select $\phi(x)=(x_{1}^{2},x_{2}^{2},\sqrt{2}x_{1}x_{2})$, then the inner product in the feature space is $x_{1}^{2}x_{1}^{'2},x_{2}^{2}x_{2}^{'2}+2x_{1}x_{2}x'_{1}x'_{2}$.
Notice that this is the same of $\langle \phi(x), \phi(x') \rangle$; thus the kernel trick consists of just using $k(x,x')=:(x^Tx')^2$.



\subsection{RKHS}
Following are the definitions that make up the basis for the theory of kernel methods.


\begin{definition}
    A sequence $\{v_n\}_{n=1}^{\infty}$ of elements of a normed space $\mathcal{V}$ is a Cauchy sequence if for every $\epsilon>0$, there exist $N=N(\epsilon) \in \mathbf{N}$ such that $\|v_n-v_m\|_{\mathcal{V}}<\epsilon \ \ \forall m,n\geq N$  
\end{definition}


\begin{definition}
    A complete metric space is a metric space in which every Cauchy sequence is convergent.
\end{definition}


\begin{definition}
    A Hilbert space is a vector space $\mathcal{H}$ with an inner product $\langle f, g \rangle$ such that the norm defined by $\|f\|=\sqrt{\langle f, f \rangle}$
turns $\mathcal{H}$ into a complete metric space.
\end{definition}

\begin{definition}
    RKHS.
    A Reproducing Kernel Hilbert Space is an Hilbert space with the evaluation functionals $\mathcal{F}_{x}(f):=f(x)$ bounded, i.e. $\forall x \in \mathcal{X}$ there exists some $C>0$ such that $\| \mathcal{F}_{x}(f)\|=\|f(x)\| \leq C \|f\|_{\mathcal{H}} \ \forall f \in \mathcal{H}$
\end{definition}


\begin{theorem}
    Riesz Representation. If $A : \mathcal{H} \rightarrow \mathbf{R}$ is a bounded linear operator in a Hilbert space $\mathcal{H}$ , there exists some $g_{A} \in \mathcal{H}$ such that $A(f) = \langle f,g_A\rangle_\mathcal{H}, \ \forall f \in \mathcal{H}$.
\end{theorem}


The Riesz representation theorem results in the following proposition for RKHS.
\begin{proposition}
For each $x \in \mathcal{X}$ there exists a function $k_{x} \in \mathcal{H}$ such that $\mathcal{F}_{x}(f)=\langle k_{x}, f\rangle_{\mathcal{H}}=f(x)$    
\end{proposition}

The function $k_{x}$ is the reproducing kernel for the point $x$.
Furthermore, note that $k_{x}$ is itself a function lying on $\mathcal{X}$

\begin{align*}
    k_{x}(y)=\mathcal{F}_{y}(k_{x})=\langle k_{x}, k_{y}\rangle_{\mathcal{H}=\langle \phi(x), \phi(x)\rangle_{\mathcal{H}}}
\end{align*}






