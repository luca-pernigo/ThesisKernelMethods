Quantile regression can be interpreted as an extension of standard regression. In this setting, you basically slice the dependent variables into quantiles and then fit a regression for each quantile. With standard regression we build a model for the conditional mean, conversely, with quantile regression we model the conditional quantile function for any desired quantile. 
Therefore, with quantile regression we are able to study the impact of covariates on quantiles directly.
\begin{definition}
    For any real valued random variable Y, we define its associated quantile function.
    \begin{equation}
        Q(q)=\inf{y:F(y)\geq q}
    \end{equation}
\end{definition}
Alternatively, in order  to ease the posing of the quantile regression problem, we can formulate quantiles as the solution to a simple optimization problem.
\\
For any $0<q<1$ consider the pinball loss function \ref{pinball} $\rho_q(u)=u(q-\mathbb{1}_{u<0})$. 
Such loss is minimized by the quantiles $Q(q)$.
We can estimate them by minimizing the expectation of $\rho_q(Y-g(x,\beta))$ with respect to the parameter $\beta$.
\\
Note, that in the special case $q=\frac{1}{2}$,  quantile regression corresponds exactly to standard regression with an absolute value loss function.
\\
It follows that the conditional linear quantile function $Q_Y(q|X=x)=x_i\beta(q)$ can be estimated by solving
\begin{equation}\label{}
    \hat{\beta(q)}=\argmin{\beta}\sum \rho_q   (y_i-x \beta)
\end{equation}
\\
Notice that, the cost function is not differentiable, therefore there is no analytical solution to the quantile regression problem. Nevertheless, we can easily solve it by employing linear programming and convex optimisation \cite{boyd2004convex}.
\\
Furthermore, can extend to non linear quantile regression by choosing a non linear model in place of $x\beta$ in the above equation.
\\
In order to illustrate the approach, let us consider applying quantile regression to the maximum daily temperature dataset in Melbourne \cite{hyndman1996estimating}; models considered will be linear, gradient boosting machine, quantile forest\cite{meinshausen2006quantile} and kernel quantile regression.
- se si puo fare plot con stessi dati ma differenti f nel quantile regression