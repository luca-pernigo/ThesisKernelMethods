%% (Master) Thesis template
% Template version used: v1.4
%
% Largely adapted from Adrian Nievergelt's template for the ADPS
% (lecture notes) project.


%% We use the memoir class because it offers a many easy to use features.
\documentclass[11pt,a4paper,titlepage]{memoir}

%% Packages
%% ========

%% LaTeX Font encoding -- DO NOT CHANGE
\usepackage[OT1]{fontenc}

%% Babel provides support for languages.  'english' uses British
%% English hyphenation and text snippets like "Figure" and
%% "Theorem". Use the option 'ngerman' if your document is in German.
%% Use 'american' for American English.  Note that if you change this,
%% the next LaTeX run may show spurious errors.  Simply run it again.
%% If they persist, remove the .aux file and try again.
\usepackage[english]{babel}
\makeatletter\AtBeginDocument{\let\@elt\relax}\makeatother

%% Input encoding 'utf8'. In some cases you might need 'utf8x' for
%% extra symbols. Not all editors, especially on Windows, are UTF-8
%% capable, so you may want to use 'latin1' instead.
\usepackage[utf8]{inputenc}

%% This changes default fonts for both text and math mode to use Herman Zapfs
%% excellent Palatino font.  Do not change this.
\usepackage[sc]{mathpazo}

%% The AMS-LaTeX extensions for mathematical typesetting.  Do not
%% remove.
\usepackage{amsmath,amssymb,amsfonts,mathrsfs}

%% NTheorem is a reimplementation of the AMS Theorem package. This
%% will allow us to typeset theorems like examples, proofs and
%% similar.  Do not remove.
%% NOTE: Must be loaded AFTER amsmath, or the \qed placement will
%% break
\usepackage[amsmath,thmmarks]{ntheorem}

%% LaTeX' own graphics handling
\usepackage{graphicx}
\usepackage{rotating}


%% We unfortunately need this for the Rules chapter.  Remove it
%% afterwards; or at least NEVER use its underlining features.
\usepackage{soul}

%% This allows you to add .pdf files. It is used to add the
%% declaration of originality.
\usepackage{pdfpages}

%% Some more packages that you may want to use.  Have a look at the
%% file, and consult the package docs for each.
\input{extrapackages}

%% Our layout configuration.  DO NOT CHANGE.
% first page settings
\setbeamerfont{title}{size=\large}
\setbeamerfont{subtitle}{size=\small}
\setbeamerfont{author}{size=\small}
\setbeamerfont{date}{size=\footnotesize}
\setbeamerfont{institute}{size=\footnotesize}
\setbeamerfont{page}{size=\footnotesize}

%set colors
\definecolor{myNewColorA}{RGB}{0, 0,0}
\definecolor{myNewColorB}{RGB}{0, 0,0}
\definecolor{myNewColorC}{RGB}{0, 0,0} % {130,138,143}

\defbeamertemplate{headline}{my header}{%
\vskip1pt%
\makebox[0pt][l]{\,\insertsection}%
\hspace*{\fill}%
\llap{\insertpagenumber\,/\,\insertpresentationendpage\,}
}
\setbeamertemplate{headline}[my header]


\setbeamercolor*{palette primary}{bg=myNewColorC}
\setbeamercolor*{palette secondary}{bg=myNewColorB, fg = white}
\setbeamercolor*{palette tertiary}{bg=myNewColorA, fg = white}
\setbeamercolor*{titlelike}{fg=myNewColorA}
\setbeamercolor*{title}{bg=myNewColorA, fg = white}
\setbeamercolor*{item}{fg=myNewColorA}
\setbeamercolor*{caption name}{fg=myNewColorA}
\usefonttheme{professionalfonts}

\usepackage{hyperref}
\hypersetup{
    colorlinks=true,
    linkcolor=black,
    filecolor=magenta,      
    urlcolor=cyan,
    pdftitle={Overleaf Example},
    pdfpagemode=FullScreen,
    }


%% Theorem environments.  You will have to adapt this for a German
%% thesis.
\input{theoremsetup}

%% Helpful macros.
\input{macrosetup}

%% Make document internal hyperlinks wherever possible. (TOC, references)
%% This MUST be loaded after varioref, which is loaded in 'extrapackages'
%% above.  We just load it last to be safe.
\usepackage[linkcolor=black,colorlinks=true,citecolor=black,filecolor=black]{hyperref}

\usepackage[symbols,nogroupskip,sort=none]{glossaries-extra}

%packages added by luca
\usepackage{subcaption}
% package for fitting big table to text width
\usepackage{adjustbox}

% url breaks
\def\UrlBreaks{\do\/\do-}

% package for multiple bibliographys
\usepackage[resetlabels,labeled]{multibib}
\newcites{W}{Sitography}

% colors
\usepackage{xcolor}

%% Document information
%% ====================

\title{Energy Market Analysis Using Kernel Methods}
\author{Luca Pernigo}
\thesistype{Master Thesis}
\advisors{Advisor: Prof.\ Dr.\ Michael Multerer
% , Dr.\ A. Eftekhari
}
\coadvisors{Co-Advisor: Dr.\ Davide Baroli }
\department{Faculty of Informatics}
\date{June 17, 2024}

\raggedbottom
\begin{document}

\frontmatter

%% Title page is autogenerated from document information above.  DO
%% NOT CHANGE.
\begin{titlingpage}
  \calccentering{\unitlength}
  \begin{adjustwidth*}{\unitlength-24pt}{-\unitlength-24pt}
    \maketitle
  \end{adjustwidth*}
\end{titlingpage}

%% The abstract of your thesis.  Edit the file as needed.


\begin{abstract}
  % introduction
  This thesis is concerned with electricity forecasting. 
  Specifically, we concentrated on the probabilistic framework. This choice was motivated by its importance in systems planning and operations as a consequence of the inception of competitive power markets, smart grids and renewable integration requirements.
  % methods
  In doing so, we focused on the family of kernel methods. We have compared them against several other statistical and machine learning techniques.
  % results
  Our results showed the feasibility of kernel methods in the field of electricity forecasting both in terms of point and probabilistic forecasting. In particular, in the probabilistic context, our experiments show the validity of kernel quantile regression equipped with the Absolute Laplacian kernel.
  % discussion
  These findings indicate that kernel methods are well suited to the characteristics of electricity.
  Anyone interested in forecasting energy should consider them when faced with the choice of the model.
  They can be employed stand-alone or combined with other valid methods into ensembles.
  

  %This example thesis briefly shows the main features of our thesis
  %style, and how to use it for your purposes.
\end{abstract}
  % The theory of kernel methods will be applied to the problem of point and probabilistic forecasting for the energy sector.
  %   Such choice is motivated by its interesting implications.


% Acknowledgements
\chapter*{Acknowledgements}
\thispagestyle{empty}

% Thank:
% - supervisor, co
First and foremost I would like to express my deepest gratitude to my advisor and co-advisor, Prof. Multerer and Dr. Baroli for their invaluable feedback, continuous support and patience during my Master's Thesis. Furthermore, I thank them for having given me the opportunity to collaborate through this thesis work.
\\
% - master director
I could not have undertaken this journey without my Master directors Prof. Schenk and Prof. Wit. I am extremely grateful for their guidance, mentorship and advice throughout my Master studies.
\\
% - professors of faculty
% From Polya How to solve it
% forster independent thinking  by challenging curiosity, stimulating questions and delving into the motives behind theory.
% tutti i professori che ho avuto, shaped my interest, saziato sete curiosità
I would like to extend my deepest appreciation to all professors I had during my studies at Usi. Their stimulating questions and push to enquire motives behind theory challenged my curiosity, shaped my interests and fostered my independent thinking.
\\
% - research group 
% discussions, suggestions
% listening the challenges I faced and insightful suggestions
I also had great pleasure of discussing with the research group of Prof. Multerer the challenges I faced during this thesis work. I thank them for their affability  and insightful suggestions.
\\
% - classmates
% friends I made along the way
% mutual support and good times we had together
% be it during a group revision session before an exam of during light hearted moments.
Special thanks to my cohort members and to the friends I made along the way. I will never forget the good times we had together.
\clearpage

%% TOC with the proper setup, do not change.
% toc=table of contents
\cleartorecto
\tableofcontents
\mainmatter

% keep listing of figures and tables
\listoffigures
\newpage
\listoftables

%% Your real content!
\chapter{Problem description}
Individuals and organizations constantly face situations of uncertainty, thus the need for robust forecasting methods. Such methods are crucial in the process of taking informed decisions and for strategic planning.
\\
The basic idea of forecasting is that we can extract knowledge from the past in order to make educated guesses about the future. Consequently, the range of fields where forecasting can be applied is very wide.
In this thesis, our focus lies on applying forecasting techniques to the energy sector. 
\\
Our decision to focus on the energy market is mainly motivated by the rapid changes it has undergone. %/experienced 
Over the last decades, electricity markets have gone through an unprecedented transformation. This shift was driven by the liberalization of such markets, the development and integration of renewable energy sources, the  increase of low carbon technologies and the adoption of smart meters. Events like the California electricity crisis are further motivating the choice of the electricity sector as subject of our studies, see \cite{california}.
Moreover, the process of deregulation lead to an increasing interest in the field of electricity forecasting (EF) within the academic community, see figure \ref{fig:epf_evolution}.
In addtion, the United Nations have identified the right to access affordable, reliable, sustainable and modern energy as one of their 17 sustainable development goals (SDGs) \citeW{un_sdgs}.
Finally, the electricity market has a set of features that make it unique: electricity cannot be stored in an efficient way and supply and demand have to be matched instantly.
\\
\section{Motivation}
There are mutliple reasons why the energy sector needs robust forecasting techniques.
For power market companies, being able to predict prices with a low mean absolute percentage error (MAPE) \ref{mape} results in increased savings \cite{savings}. Furthermore, the adoption of smart meters provides power market companies with a huge amount of consumer data. This enables them to better model consumer preferences.
\\
Transmission system operators' (TSO) main goal is to match supply and demand, generally TSO do so by increasing or decreasing the generation. Thus, from their point of view forecasting is critical for balancing the electricity network.
Probabilistic forecasting may be useful to power producers, traders and consumers in order to improve their decision making process and managing risk. This holds in particular for traders, because probabilistic forecasts enable them to simulate scenarios and carry out stress tests.
Other possible applications include: control of storage, demand side response, anomaly detection, network design and planning, simulating inputs and handling missing data.
\\
\section{Point versus probabilistic forecast}
A distinction has to be made between two types of forecasting approaches: point forecasts and probabilistic forecasts.
Point prediction, also called deterministic forecasting in the literature \cite{EPF_review}, is all about predicting a particular value in time.
On the other hand, with probabilistic forecasting we aim at predicting either an interval, quantiles or a probability distribution for each point in time \cite{nowotarski}. For this reason, probabilistic forecasts are more informative than point forecasts. This is why the interest of the research community is shifting towards them.
A probabilistic forecast can be turned into a point forecast by simply taking its expectation.
Alternatively, a probabilistic forecast can be derived from a point one by modeling the residuals of the point prediction.



\section{Aims and objectives}

% - what is the question? objectives of the thesis
% the description of the problem tackled and the methodology used to solve
% it.

The scope of the thesis is analyzing state of the art forecasting methods in the energy market and to compare and to integrate them with ideas coming from the theory of kernel methods.  



\section{Outline}
We start with a literature review and a bibliometric analysis in Section \ref{literature_review}.
Then, the theory underlying kernel methods is covered in Section \ref{kernel_theory}. Evaluation metrics necessary to rank the forecasting techniques are presented in Section \ref{metrics}. Section \ref{energy_market} explains the core features and terminology of the energy market and of the electricity newtork.
Following, Sections \ref{ch:point} and \ref{ch:prob} introduce the state of the art methods in the context of point and probabilistic forecasting respectively.
Section \ref{eda} goes on with the extract-load-transform (ETL) pipeline and the exploratory data analysis (EDA). 
Implementation details are included in Section \ref{implementation}.
Finally, Section \ref{analysis} presents experiments results and discusses models' strenghts, weaknesses and possible improvements.

\chapter{Literature review}\label{literature_review}
Energy Forecasting 

\subsection{Energy forecasting classification}

\subsubsection{Types}
In the context of energy forecasting, the quantities of most interest are price, load and renewables generation.

\subsubsection{Forecasting horizons}
\subsubsection{Size}
- Forecasting horizons

- Refer to section \ref*{metrics} for evaluating metrics

- plots come quelli delle review con i dati di Scopus se é una cosa fattibile
\label{sec:literaturereview}

\begin{figure}[h!]
    \includegraphics[width=\textwidth]{images/epf_evolution1.jpg}
    \caption{EPF publications \cite{EPF_review}}
    \label{fig:epf_evolution}
  \end{figure}

\begin{figure}[h!]
  \includegraphics[width=\textwidth]{images/point_vs_prob.jpg}
  \caption{Point vs probabilistic pubblications}
  \label{fig:point_vs_prob}
\end{figure}

\begin{figure}[h!]
  \includegraphics[width=\textwidth]{images/cs_stat_both.jpg}
  \caption{Pubblications by method}
  \label{fig:point_vs_prob}
\end{figure}


\begin{figure}[h!]
  \includegraphics[width=\textwidth]{images/subject_area.jpg}
  \caption{Pubblications by subject area}
  \label{fig:subject_area}
\end{figure}


\begin{figure}[h!]
  \includegraphics[width=\textwidth]{images/src_title.jpg}
  \caption{Most popolar sources/outlets}
  \label{fig:src_title}
\end{figure}

%INTRO KERNEL METHODS E ENERGY PRICE MODELLING

Kernel methods are a class of algorithms for patter analysis.
With kernel methods we are able to apply linear methods with predictors in a high dimensional space, without having to explicitly evaluate the involved dot products of the features.
In this thesis work I will address the performance of kernel methods in the context of probabilistic forecasting; the area of application will be the electricity market. 
Probabilistic forecasting may be useful to power producers, traders and consumers in order to improve their decision making process and managing risk(VaR). This is because probabilistic forecast enables them to simulate scenarios and carry out stress tests.


%SPIEGO UN PO' I PAPER CHE HO FATTO PASSARE VELOCEMENTE 
Every paper uses different datasets (heterogeneous)
So it is not possible to compare directly results
from one paper to another without implementing the
paper specific algorithms and the applying them to
your dataset.
This is why sections after are destined to analyzing
how these proposed methods so far work and their mathematical
theory details
\\
This problem has been already addressed in \cite{probablistic_electricity_forecast} and \cite{probablistic_electricity_forecast2}. 
These papers surveyed the performance of neural network architectures against simpler approaches like quantile regression and data fitting to Johnson distribution. Their conclusion is that distributional NN perform a little worse than quantile regression but the former has smaller computational cost; that is because the quantile regression is run for every quantile from 0.01 to 0.99.
Nevertheless kernel methods received very little attention in this specific setting.
\\
Kernel methods considered are kernel mean embedding \cite{pmlr}, \cite{Muandet_2017} and kernel herding \cite{supersamples}. Particularly extending the idea of \cite{2022nystrom}, where the Nyström approximation is employed in computing the kernel mean embedding, experiments with the Pivoted Cholesky decomposition will be performed.
\\
In section 2, three papers that lay the basis for this thesis’ work are summarized.
\\
************************
\\
During the past 25 years %because paper of 2014 says 15, so now it is 25
a wide range of new ideas have been proposed for point forecasting and for probabilistic forecasting.
\\
The field benefitted greatly from the increse of computing power, the greater availability of dataset and the interest in data science.
As a consequence, the forecaster's toolbox has grown in size and complexity.
\\
Such variety of methods is characterized by heterogeneity in the fields from which they come from; methods come from statisitics, mathematics, econometrics, electrical engineering and the artificial intelligence communities.
\\
Before delving into the literature review, it is important to make clear that at this point in time there is no superior method. Different solutions may outperform or underperform compared to other techniques depending on the problem settings. Thus, understanding the complexity, strenghts and weaknesses of each method is crucial for fitting the right model to the right setting.
\\
Finally, within this research community, emerged the need of more homogeneity in the choice of the error valuation metrics and in the way of comparing model performances \cite{EPF_review}.
\\
************************
\\
Lately, the idea of combining forecasts has gained popularity in the forecasting communit \cite{forecasting_big}; in the literature, combined forecasts are called ensemble \cite{gneiting_weather_ensemble}.
Experimental results have shown ensemble methods to outperform their component forecasts.
\\
Note that the more the errors of the combined models are not correlated the more we can benefit from ensembles.
\\
It is also worth noting that older and simpler methods are still valuable(in combination with other models or on their own); these being less subject to overfitting than complex models.


% WORKPLAN from Fall semester
% \chapter{Workplan}
% Workplan is to start by considering the performance of kernel herding in generating an empirical distribution of the electricity prices and compare the performance to the methods in \cite{probablistic_electricity_forecast}.
To do so, with kernel herding we generate samples and as a consequence we also have an empirical distribution of electricity prices. Then we take the quantiles from 0.01 to 0.99 and finally we evaluate the distribution forecast through the Continous Ranked Probability Score CRPS.
\\
In this study, the data from the \href{https://www.epexspot.com/en/market-data?market_area=CH&trading_date=2023-12-11&delivery_date=2023-12-12&underlying_year=&modality=Auction&sub_modality=DayAhead&technology=&product=60&data_mode=table&period=&production_period=}{EPX} martket will be used; data is retrieved daily from the data provider through an automatic python script.
\\
Depending on the results of kernel herding, we may also consider how kernelized quantile regression behaves in the same setting. Next, additional kernel methods applied to problems concerning energy prices and related subjects could be taken into account.


% this introduction file explains how to use the 
% provided template
% \input{introduction}

\chapter{Kernel methods}\label{kernel_theory}
% Some commands used in this file

\section{Kernel mean embedding of distributions: A review and beyond}
% \href{https://arxiv.org/abs/1605.09522}{
From this first paper \cite{Muandet_2017}, the notation and terms used in the theory of Reproducing Kernel Hilbert Spaces are summarized.
\\
Many algorithms use the inner product as similarity measure between data instances $x, x' \in \mathcal{X}$. However, this inner product spans only the class of linear similarity measures. 
\\
The idea behind kernel methods is to apply a non-linear transformation $\phi$ to the data $x$ 
in order to get a more powerful non linear similarity measure.
\begin{align*}
\phi(x):\mathcal{X} &\longrightarrow \mathcal{F}
    \\
    x&\mapsto \phi(x)
\end{align*}


Then we take the inner product in the high dimensional space $\mathcal{F}$ mapped by $\phi(x)$.
\\
\begin{align*}
k(x,x'):=&\langle \phi(x), \phi(x') \rangle_{\mathcal{F}}
\end{align*}
\\
$\phi(x)$ is referred as feature map while $k$ as kernel function.
\\
Therefore, we can kernelize any algorithm involving a dot product by substituting $\langle x, x' \rangle_{\mathcal{X}}$ with $\langle \phi(x), \phi(x') \rangle_{\mathcal{F}}$
\\
One would expect constructing the feature maps explicitly and then evaluate their inner product in $\mathcal{F}$ to be computationally expensive, and indeed it is. However, we do not have to explicitly perform such calculations. This is because of the existence of the kernel trick.
To illustrate the idea behind the kernel trick consider the following example. 
\\
Suppose $x \in \mathbb{R}^2$ and assume to select $\phi(x)=(x_{1}^{2},x_{2}^{2},\sqrt{2}x_{1}x_{2})$, then the inner product in the feature space is $x_{1}^{2}x_{1}^{'2},x_{2}^{2}x_{2}^{'2}+2x_{1}x_{2}x'_{1}x'_{2}$.
Notice that this is the same of $\langle \phi(x), \phi(x') \rangle$; thus the kernel trick consists of just using $k(x,x')=:(x^Tx')^2$.



\subsection{RKHS}
Following are the definitions that make up the basis for the theory of kernel methods.


\begin{definition}
    A sequence $\{v_n\}_{n=1}^{\infty}$ of elements of a normed space $\mathcal{V}$ is a Cauchy sequence if for every $\epsilon>0$, there exist $N=N(\epsilon) \in \mathbf{N}$ such that $\|v_n-v_m\|_{\mathcal{V}}<\epsilon \ \ \forall m,n\geq N$  
\end{definition}


\begin{definition}
    A complete metric space is a metric space in which every Cauchy sequence is convergent.
\end{definition}


\begin{definition}
    A Hilbert space is a vector space $\mathcal{H}$ with an inner product $\langle f, g \rangle$ such that the norm defined by $\|f\|=\sqrt{\langle f, f \rangle}$
turns $\mathcal{H}$ into a complete metric space.
\end{definition}

\begin{definition}
    RKHS.
    A Reproducing Kernel Hilbert Space is an Hilbert space with the evaluation functionals $\mathcal{F}_{x}(f):=f(x)$ bounded, i.e. $\forall x \in \mathcal{X}$ there exists some $C>0$ such that $\| \mathcal{F}_{x}(f)\|=\|f(x)\| \leq C \|f\|_{\mathcal{H}} \ \forall f \in \mathcal{H}$
\end{definition}


\begin{theorem}
    Riesz Representation. If $A : \mathcal{H} \rightarrow \mathbf{R}$ is a bounded linear operator in a Hilbert space $\mathcal{H}$ , there exists some $g_{A} \in \mathcal{H}$ such that $A(f) = \langle f,g_A\rangle_\mathcal{H}, \ \forall f \in \mathcal{H}$.
\end{theorem}


The Riesz representation theorem results in the following proposition for RKHS.
\begin{proposition}
For each $x \in \mathcal{X}$ there exists a function $k_{x} \in \mathcal{H}$ such that $\mathcal{F}_{x}(f)=\langle k_{x}, f\rangle_{\mathcal{H}}=f(x)$    
\end{proposition}

The function $k_{x}$ is the reproducing kernel for the point $x$.
Furthermore, note that $k_{x}$ is itself a function lying on $\mathcal{X}$

\begin{align*}
    k_{x}(y)=\mathcal{F}_{y}(k_{x})=\langle k_{x}, k_{y}\rangle_{\mathcal{H}=\langle \phi(x), \phi(x)\rangle_{\mathcal{H}}}
\end{align*}




\subsection{Kernel families}
Table \ref{tab:kernel types} contains popular kernel families in literature and applications.
\begin{table}
    \caption{Kernel types}
    \begin{tabular}{lll}
        \toprule
       Kernel function & Equation & Hyperparameters \\
       \midrule
       Linear &  $k(x_1,x_2)=x_1x_2$ &   \\
       Polynomial &  $k(x_1,x_2)=(x_1^\intercal x_2+c)^d$ &   c, d\\
       Gaussian RBF &  $k(x_1,x_2)=e^{-\frac{\|x_1-x_2|\|^2}{2\sigma^2}}$ &   $\sigma$\\
       Exponential RBF/Laplacian& $k(x_1,x_2)=e^{-\frac{|x_1-x_2|}{\gamma}}$ &   $\gamma$\\
       Hyperbolic/Sigmoid Kernel &  $k(x_1,x_2)=\tanh(\gamma x_1^\intercal x_2+r)$ &  $\gamma$, r \\
       Periodic &  $k(x_1,x_2)=e^{\frac{-2 sin^2\left(\frac{\pi}{p}|x_1-x_2| \right)}{l^2}} $ &   p, l\\
        \bottomrule
    \end{tabular}
    \label{tab:kernel types}
\end{table}


% \section{Recovering distributions from Gaussian RKHS embeddings}
% \href{http://proceedings.mlr.press/v33/kanagawa14.pdf}
This paper covers the RKHS embedding approach to nonparametric statistical inference \cite{pmlr}.
The idea here is computing an estimate of the kernel mean in order to obtain an approximation of the underlying distribution of the observed random variable.
The kernel mean embedding $\mu_{\mathbb{P}}$ of a probability $\mathbb{P}$ corresponds to the feauture map $\phi(x)$ integrated with respect to the $\mathbb{P}$ measure.
\\
That is $\mu_{\mathbb{P}}:=\int_{\mathcal{X}}k(x,\cdot)d\mathbb{P}(x)$.
\\
Kernel mean embedding serves as a unique representation of $\mathbb{P}$ in the RKHS $\mathcal{H}$. This holds provided  that $\mathcal{H}$ is characteristic. 
\begin{definition}
The RKHS $\mathcal{H}$ and the associated kernel k are said characteristic, when the mapping $\mu : \mathbb{P} \rightarrow \mathcal{H}$ is injective.
\end{definition}
%Hence, H is characteristic, if for any P,Q ∈ P, we have μP = μQ if and only if P = Q. 
When the mapping is injective, we have that $\mu_{\mathbb{P}}$ is uniquely associated with $\mathbb{P}$; thus, $\mu_{\mathbb{P}}$ is a unique representation of $\mathbb{P}$ in $\mathcal{H}$. 
%Characteristic kernels on X = Rd include {Gaussian, Matérn and Laplace}(Sriperumbudur et al., 2010).
%On the other hand, linear and polynomial kernels are not characteristic.
\\
Note that, by the reproducing property of RKHS $\langle f, k(x,\cdot) \rangle=f(x)$ we have:
\\
\begin{center}
    $E_{\mathbb{P}}[f(x)]=\langle f,\mu_{\mathbb{P}} \rangle_{\mathcal{H}}, \ \forall f \in \mathcal{H}$
\end{center}

\begin{proof}   
\begin{align*}
E_{\mathbb{P}}[f(x)]&=\int_{\mathcal{X}} f(x) d\mathbb{P}(x)\\
&=\int_{\mathcal{X}} \langle f, k(x,\cdot) \rangle_{\mathcal{H}} d\mathbb{P}(x)\\
&= \sum_{i=1}^{\infty} \langle f, k(x_{i}, \cdot) \rangle_{\mathcal{H}} \mathbb{P}(\mathcal{X}_{i}) \\
&=\langle f, \sum_{i=1}^{\infty} k(x_{i}, \cdot)\mathbb{P}(\mathcal{X}_{i}) \rangle_{\mathcal{H}}\\
&=\langle f, \mu_{\mathbb{P}} \rangle_{\mathcal{H}}
\end{align*}
\end{proof}

The kernel mean embedding can be employed to estimate the density $p$ at any fixed point $x_{0}$. Letting $\delta_{x_{0}}$ to be the dirac delta function we have:
\\
\begin{center}
$p(x_{0})=\int \delta_{x_{0}}p(x)dx=E_{\mathbb{P}}[\delta_{x_{0}}]$    
\end{center}

Therefore, the idea is to define an estimator for the expectation of $\delta_{x_{0}}$ through $\mu_{\mathbb{P}}$; this would result in an estimator of $p(x_{0})$.
\\
A kernel $k(x_{0},\cdot)$ is used to approximate the delta function, furthermore applying theorem 1 of \cite{pmlr} we have that a consistent estimator of $E_{\mathbb{P}}[k(x_{0}, \cdot)]$ is given by $\sum\limits_{i=1}^{n}w_{i}k(x_{0},x_{i})$
\\
When the weigths are all $1/n$ we end up with the standard kernel density estimation.
\\
Alternatively,  the optimal weights can be found by minimizing the following problem $ \| \hat{\mu}-\Phi w\|^2$ 
where $\Phi:\mathbb{R}^{n} \rightarrow \mathcal{H}$.
%, \alpha \rightarrow \sum\limits_{j=1}^{n}\alpha_{j} \phi(x_{j})$
\\
% \section{\href{https://arxiv.org/pdf/1203.3472.pdf}{Super-Samples from Kernel Herding}}

Kernel herding is a deterministic sampling algorithm designed to draw "Super Samples" from probability distributions \cite{supersamples}. The idea of herding, is to generate pseudo-samples that greedily minimize the error betweeen the mean operator and the empirical mean operator resulting from the selected herding points.
% note the kernel mean embedding lies in a.
%\\

%Let $x \in \mathcal{X}$, where a general $x$ is a state over the index set $\mathcal{X}$. Note, $\mathcal{X}$ is typically the space of covariates.

%\\
%Let $\phi:\mathcal{X}\rightarrow \mathcal{H}$ denote a feature map into a Hilbert space $\mathcal{H}$ endowed with inner product $\langle \cdot, \cdot \rangle_{\mathcal{H}}$
Letting $p(x)$ be a probability distribution, kernel herding is recursively defined as follows:
\begin{align*}
x_{t+1} &=\argmax{x\forall \mathcal{X}}\langle w_{t},\phi(x)\rangle 
\\
w_{t+1} &= w_{t} + E_{\mathbb{P}}[\phi(x)]-\phi(x_{t+1})    
\end{align*}
\\
$w$ denotes a weight vector that lies in $\mathcal{H}$.
\\
Here, by assuming that the inner product between weights and the mean operator is equal to a general functional f evaluated at x, that is $\langle w, \phi(x) \rangle_{\mathcal{H}}=f(x)$.
We have:
\begin{align*}
\langle w, \mu_{\mathbb{P}} \rangle_{\mathcal{H}}
&= \langle w, \int k(x,\cdot)d\mathbb{P}(x) \rangle_{\mathcal{H}}
\\
&= \langle w, \sum\limits_{i=1}^{\infty} k(x_{i},\cdot)\mathbb{P}(\mathcal{X}_{i})\rangle_{\mathcal{H}}
\\
&= \sum \limits_{i=1}^{\infty} \langle w, k(x_{i}, \cdot) \rangle_{\mathcal{H}}\mathbb{P}(\mathcal{X}_{i})
\\
&= \sum \limits_{i=1}^{\infty}f(x_{i})\mathbb{P}(\mathcal{X}_{i})
\\
&=\int f(x) d\mathbb{P}(x)
\\
&= \mathbb{E}_{\mathbb{P}} [f(x)]
\end{align*}
\\
Moreover, second assumption of the model is that $\|\phi(x)\|_{\mathcal{H}}=R \quad \forall x \in X$.
That is the Hilbert space norm of the feature vector is equal to a constant R for all states in the set $\mathcal{X}$.
\\
\\
This can be achieved by taking the new feature vector as $\phi^{new} (x)=\frac{\phi(x)}{\|\phi(x)\|_{\mathcal{H}}}$.
See \ref{appendix:new_feature} for details.
\\
By rewriting the formula for the weights we end up with
\begin{align*}
w_{t+1} &= w_{t} + E_{\mathbb{P}}[\phi(x)]-\phi(x_{t+1})
\\
&= w_{t-1} + E_{\mathbb{P}}[\phi(x)]+E_{\mathbb{P}}[\phi(x)]-\phi(x_{t+1})-\phi(x_{t})
\\
&= w_{t-2} + E_{\mathbb{P}}[\phi(x)]+E_{\mathbb{P}}[\phi(x)]+E_{\mathbb{P}}[\phi(x)]-\phi(x_{t+1})-\phi(x_{t})-\phi(x_{t-1})
\\
&\text{Considering $w_{T}$, we have}
\\
&
w_{T}=w_{0}+TE_{\mathbb{P}}[\phi(x)]-\sum\limits_{t=1}^{T}\phi(x_{t})
\end{align*}
Note that
$E_{\mathbb{P}}[\phi(x)]=\int_{\mathcal{X}}\phi(x)d\mathbb{P}(x)$ which corresponds to the definition of $\mu_\mathbb{P}$. 
That is $\mu$ is the mean operator associated with the distribution $\mathbb{P}$; it lies in $\mathcal{H}$.
\\
Thus,
\begin{align*}
   w_{T}= w_{0}+T\mu_{\mathbb{P}}-\sum\limits_{t=1}^{T}\phi(x_{t})
\end{align*}
\\
Notice we do not have to compute $\mu_{\mathbb{P}}$  explicitly, the terms involving $\mu_{\mathbb  {P}}$ will be computed by applying the kernel trick.
\\
% In addition we can construct an estimate $\hat{\mu}=:\frac{1}{N}\sum \limits_{i=1}^{N} k(x_{i},\cdot)$.
\\
Now we have everything we need in order to reformulate the original problem in a way such that it depends just on the states x.
Plug the formula for the weights in the formula for the $x_{t}$ and use the kernel trick; we end up with
\\
\begin{align*}
    x_{T+1} &=\argmax{x\forall \mathcal{X}}\langle w_{0}+T\mu_{\mathbb{P}}-\sum\limits_{t=1}^{T}\phi(x_{t}),\phi(x)\rangle_{\mathcal{H}} \\
&=\argmax{x\forall \mathcal{X}}\langle w_{0},\phi(x)\rangle_{\mathcal{H}}
+\langle T\mu_{\mathbb{P}},\phi(x)\rangle_{\mathcal{H}}
-\langle \sum\limits_{t=1}^{T}\phi(x_{t}),\phi(x)\rangle_{\mathcal{H}}
\\
&=\argmax{x\forall \mathcal{X}}\langle w_{0},\phi(x)\rangle_{\mathcal{H}}
+T\langle \mu_{\mathbb{P}},\phi(x)\rangle_{\mathcal{H}}
- \sum\limits_{t=1}^{T}k(x_{t}, x)
\end{align*}
\\
Notice $\langle \mu_{\mathbb{P}},\phi(x)\rangle_{\mathcal{H}}$ can be rewritten in the following way
\\
\begin{align*}
\langle \mu_{\mathbb{P}},\phi(x)\rangle_{\mathcal{H}}&=
\langle \int_{\mathcal{X'}}\phi(x')d\mathbb{P}(x'),\phi(x)\rangle_{\mathcal{H}}
\\
&=
\langle  \sum\limits_{i=1}^{\infty}\phi(x'_{i})\mathbb{P}(\mathcal{X'}_{i}),\phi(x)\rangle_{\mathcal{H}}
\\
&=
\sum\limits_{i=1}^{\infty}\langle \phi(x'_{i}),\phi(x)\rangle_{\mathcal{H}} \mathbb{P}(\mathcal{X'}_{i})
\\
&=
\sum\limits_{i=1}^{\infty}k(x'_{i},x) \mathbb{P}(\mathcal{X'}_{i})
\\
&=
\int_{\mathcal{X'}}k(x',x)d\mathbb{P}(x')
\\
&=
E_{\mathbb{P}}[k(x',x)]
\end{align*}
\\
% Note the introduction of $x'$ is the same as as $x$ it is just notation.
\\
Furthermore, by initializing $w_{0}=\mu_{\mathbb{P}}$ we end up with the following function to be optimized, i.e.
\\
\begin{align*}
x_{T+1}&=\argmax{x\forall \mathcal{X}}\langle w_{0},\phi(x)\rangle +T\langle \mu_{\mathbb{P}},\phi(x)\rangle- \sum\limits_{t=1}^{T}k(x_{t}, x)    
\\
&=\argmax{x\forall \mathcal{X}} (T+1)E_{\mathbb{P}}[k(x',x)]- \sum\limits_{t=1}^{T}k(x_{t}, x)    
\end{align*}
\\ 
Now consider the error term between the mean kernel operator and its estimation through herding samples
\\
\begin{align*}
\varepsilon_{T+1}&=\| \mu_{\mathbb{P}} -\frac{1}{T+1}\sum\limits_{i=1}^{T+1} \phi(x_{t})\|_{\mathcal{H}}^{2}
\\
&=\mathbb{E}_{x,x' \sim \mathbb{P}} [k(x',x)] - \frac{2}{T+1}\sum \limits_{t=1}^{T+1} \mathbb{E}_{x \sim \mathbb{P}} [k(x,x_{t})] +\frac{1}{(T+1)^2}\sum \limits_{t,t'=1}^{T+1} k(x_{t}, x_{t'})
\\
&=
\mathbb{E}_{x,x' \sim \mathbb{P}} [k(x',x)] - \frac{2}{T+1}\sum \limits_{t=1}^{T+1} \mathbb{E}_{x \sim \mathbb{P}} [k(x,x_{t})] +\frac{1}{(T+1)^2}\sum \limits_{\substack{
t=1\\ t=t'}}^{T+1} k(x_{t}, x_{t'})+
\\
&\hspace{0.5cm}
+\frac{2}{(T+1)^2}\sum \limits_{\substack{
t=1\\ t\neq t'}}^{T+1} k(x_{t}, x_{t'})
\\
\end{align*}
So $\varepsilon_{T+1}$ depends on $x_{T+1}$ only through $-\frac{2}{T+1} \mathbb{E}_{x \sim \mathbb{P}} [k(x,x_{T+1})]+
\frac{2}{(T+1)^2} \sum \limits_{t=1}^{T}k(x_{t},x_{T+1})
$
The term $k(x_{T+1},x_{T+1})$ is not included, because by assumption it is equal to the constant R.
\\
Recognize that this term is the negative of the objective function maximized with respect to x. So the sample $x_{T+1}$ minimizes the error at time step $T+1$, i.e. $\varepsilon_{T+1}$ 
%Notice that for many kernels, explicit expressions for ${E}_{x' \sim \mathbb{P}} [k(x,x')]$ have been obtained, see \cite{jebara}.
\\
During the iterative step of kernel herding we maximize the negative of this quantity, thus we are minimizing the error greedily. In the sense that at each iteration we choose the x that minimizes our current error; however this does not guarantee that the samples states are jointly optimal.
\\
Intuitively, at each iteration, herding searches for a new sample to add to the pool; it is attracted to the regions where p is high and pushed away from regions where samples have already been selected.
\\
% At each iteration, this algorithm searches for a new datapoint to add to the set of supersamples. After that, kernel mean embedding using the supersamples can be computed.
% Additionally, note that the kernel herding algorithm can be used to generate samples from the learned distribution that are more informative then i.i.d. random samples.
% %\section{\href{https://deliverypdf.ssrn.com/delivery.php?ID=114069009009097116025022100083013106004049020088012091073068103111009120067008111024018110017063062049097112002108029126122000008043088026052070110126094088002122105027052007110064122001007119000081092004068099090094101088105091029096125122031094029066&EXT=pdf&INDEX=TRUE}{Multivariate probabilistic forecasting of electricity prices with trading applications}}

% \newpage
% - Explain kernel density estimation
Explain under which conditions kernel mean embedding
is equivalent to kernel density estimation.

- Other kernel theory concepts, I may need to
restructure the structure of the kernel folder
by putting in the right order the varies paper1,2,3,4



\chapter{Evaluation metrics}\label{metrics}
Proper evaluation methods guide researchers in choosing the model that best fits their needs; thus, this chapter is dedicated to the most common evaluation metrics adopted by academics in the field of electricity forecasting. Error metrics and measures vary depending on whethter we are concerned with point or probabilistic forecasts. Additionally, note that the latter can take different forms which therefore requires different measures.
\section{Mean absolute error}\label{mae}
Consider the time series with actual values given by $y=(y_{n+1}, y_{n+2},\dots, y_{n+h})$
and its h step ahead point forecast $\hat{y}=(\hat{y}_{n+1}, \hat{y}_{n+2},\dots, \hat{y}_{n+h})$ the mean absolute error (MAE) is defined as
\begin{definition}
    $\mathrm{MAE}(y,\hat{y})=\frac{1}{h}\| y- \hat{y}\|_{1}=\frac{1}{h}\sum\limits_{k=1}^{h}|y_{n+k}-\hat{y}_{n+k}|$
\end{definition}

\section{Root mean squared error}\label{rmse}
\begin{definition}
    $\mathrm{RMSE(y, \hat{y})}=\frac{1}{\sqrt{h}}\|y-\hat{y}\|_{2}=\sqrt{\frac{\sum\limits_{k=1}^{h}(y_{n+k}- \hat{y}_{n+k})^2}{h}}$
\end{definition}

Mean absolute error and root mean squared error posses the useful property of being expressed in the same units of the data, thus enabling meaningful comparisons.
However, a drawback of such measures is that we cannot use them to compare accuracy between time series which have different magnitudes. For instance, a day ahead error of 1kWh is negligible when considering a daily demand of 100kWn while the same error is considerably big when daily demand is 2kWh. This consideration leads to relative accuracy scores, between those the mean absolute percentage error (MAPE) is by far the most popular.


\section{Mean absolute percentage error}\label{mape}. 
\begin{definition}
    $\mathrm{MAPE}(y,\hat{y})=\frac{100}{h}\sum\limits_{k=1}^{h}\frac{|y_{n+k}-\hat{y}_{n+k}|}{|y_{n+k}|}$
\end{definition}

\section{Root mean squared percentage error}\label{rmspe}
\begin{definition}
    $\mathrm{RMSPE}(y,\hat{y})=100\cdot\sqrt{\frac{1}{h}\sum\limits_{k=1}^{h} \left(\frac{|y_{n+k}-\hat{y}_{n+k}|}{|y_{n+k}|}\right)^2}$
\end{definition}
Note, mean absolute percentage error and root mean squared percentage error may not be appropriate for series which have zero or very small values, for example, electricity demand at the household level; the result is a large score regardless of the absolute errors.
Scaled errors constitute a robust family of scores.
\section{Mean absolute scaled error}\label{mase}
\begin{definition}
    $\mathrm{MASE}(y,\hat{L})=\frac{1}{h}\sum\limits_{k=1}^N\frac{|y_{n+k}-\hat{y}_{n+k}|}{\frac{1}{h-1}\sum\limits_{k=2}^{h}|y_{k}-y_{k-1}|}$
\end{definition}
In the denominator we have the error of the naïve/persistence model. 
In this model, the current demand makes up the prediction for the next time step; that is $\hat{y}^{\mathrm{naive}}_{n+1}=y_{n}$.
\section{Root mean squared scaled error}\label{rmsse}
\begin{definition}
    $\mathrm{RMSSE}(y,\hat{y})=\sqrt{\frac{1}{h}\sum\limits_{k=1}^N\left(\frac{|y_{n+k}-\hat{y}_{n+k}|}{\frac{1}{h-1}\sum\limits_{k=2}^{h}|y_{k}-y_{k-1}|}\right)^2}$
\end{definition}


\section{Pinball}\label{pinball}
The pinball score or quantile score is used to measure the accuracy of a quantile forecast.
\begin{definition}
    $\mathrm{Pinball}(y_{t},\hat{y}_{t,q},q)=
\begin{cases}
(q-1)(y_{t}-\hat{y}_{t,q}) & y_t > \hat{y}_{t,q} \\
q(\hat{y}_{t,q}-y_t) & y_t \leq \hat{y}_{t,q}
\end{cases}$
\end{definition}
The pinball loss is an asymmetric function, it weights its score differently depending on the error sign and on the quantile considered, see figure \ref{fig:pinball}.
\begin{figure}
    \includegraphics[width=\textwidth]{images/pinball_loss.png}
    \caption{Pinball loss}
    \label{fig:pinball}
  \end{figure}
By averaging all the pinball losses over all quantiles and over the whole forecast horizon, we obtain the pinball loss of the probabilistic forecast.
\section{Winkler}
\begin{definition}
    $\mathrm{Winkler}(y_t, \hat{y}_t,q)=\begin{cases}
        \delta & l_{t}\leq y_{t}\leq u_{t}\\
        \delta+2(l_{t}-y_{t})/\alpha & y_{t}< l_{t}\\
        \delta+2(l_{t}-u_{t})/\alpha & y_{t} \geq u_{t}
    \end{cases}$
\end{definition}
Where delta is the prediction interval (PI) width, that is $\delta=b_t-a_t$ where $u$ is the prediction interval upper threshold and $l$ is the prediction interval lower threshold. This score penalizes observations falling outside the prediction interval and rewards narrow prediction intervals.
\section{Continous ranked probability score}
The continous ranked probability score (CRPS) measures the differece between the estimated cumulative distribution $\hat{F}$ and the empirical cumulative density function (CDF).
\begin{definition}\label{def_crps}
    $\mathrm{CRPS}(y, \hat{F})=\int\limits_{-\infty}^{\infty}\left(\hat{F}(x)-\mathbb{1}(x-y) \right)^2 dx$
\end{definition}
% Nevertheless, we can evaluate the integral in closed form. 
Where the indicator function is defined as 
    $\mathbb{1}(z)=
\begin{cases}
0 & z<0\\
1 & z \geq 0
\end{cases}$
\\
For a visualisation see, \ref{fig:crps}. The grey area is what contributes toward the CRPS score.
The better the estimated cumulative density function is the smaller the total CRPS score will be.
\begin{figure}
    \includegraphics[width=\textwidth]{images/crps.png}
    \caption{CRPS integral \cite{haben2023core}}
    \label{fig:crps}
  \end{figure}
  \\
It is worth noting, that the CRPS integral can be rewritten in terms of expectations. This makes its evaluation easier, since we know that the sample mean converges to the expectation by the law of large numbers. This was first pointed out by \cite{proper_scores}, where authors take advantage of lemma 2.2 of \cite{new_multi_test2} or equivalently identity 17 of \cite{new_multi_tes1}.
\\
\begin{lemma}
    Let $X_1$, $X_2$, $Y_1$, $Y_2$ be independent real random variables with finite expectations. Let $X_1,X_2$ be identically distributed with distribution function $F$ and let $Y_1,Y_2$ be identically distributed with distribution function $G$. Then
   \begin{equation}
    \mathbb{E}(|X_1-Y_1|)-\frac{1}{2}\mathbb{E}(|X_1-X_2|)-\frac{1}{2}\mathbb{E}(|Y_1-Y_2|)=\int\limits_{-\infty}^{\infty}\left(F(x)-G(x)\right)^2dx
\end{equation}
\end{lemma}
Notice that, in our case, equation \ref{def_crps}, the distribution $G$ of $Y_1$ and $Y_2$ is degenerate, with all probability mass on a single point.
%  (x notation will need to be cleaned here). Since $G(t)=:$
It follows that, the third addend in the summation is zero. That is because the fact of $Y_1$ and $Y_2$ both following distribution $G$ implies that $\mathbb{E}(|Y_1-Y_2|)$ corresponds to the difference of two equal constant numbers.
\\
Additionally, since $Y_1$ is just a constant, we have $Y_1=y$.
\\
Putting everything together we have obtained an alternative way of computing the CRPS score.
\begin{equation}
    \int\limits_{-\infty}^{\infty}\left(\hat{F}(x)-\mathbb{1}(x-L) \right)^2 dx=\mathbb{E}(|X_1-y|)-\frac{1}{2}\mathbb{E}(|X_1-X_2|)
\end{equation}
\\
\section{Probability integral transform}
The probability integral transform (PIT) is a method to assess visually the quality of probabilistic a forecasts. PIT is obtained by applying the predicted cumulative density function $\hat{F}$ to your data; if applying such CDF to the data results in a uniform distributed PIT, then $\hat{F}$ is a valid prediction. If not, $\hat{F}$ is not the suited CDF for the considered data. Figure \ref{fig:pit} provides an example, applying the true CDF results in a well calibrated PIT (left). Alternatively, applying a bad CDF results in either a overdispersed (middle) or underdispersed (right) PIT.
\begin{figure}
    \includegraphics[width=\textwidth]{images/pit.png}
    \caption{PIT types \cite{haben2023core}}
    \label{fig:pit}
  \end{figure}
\\
%RELIABILITY PLOT
%PICP,NMPIW,CWC,unconditional coverage(arora)-->Probability density forecasting of wind speed based on quantile regression and kernel density estimation Lei Zhang

% Reliability and sharpness \section{Criteria}-->Recent advances in electricity price forecsating a review of probabilistic forecasting Weron
% Two criteria are used in literature to evaluate probabilistic forecasting: sharpness and reliability. Reliability means 

\chapter{The energy market}\label{energy_market}
Chapter explaining the energy market

- How auctions work

- Difference intraday dayhead all other stuff

- Prices can be negative


\chapter{Point forecasting}\label{ch:point}
This chapter covers the theory of the most widely used method for point forecasting in the EF literature. Besides discussing the theory underlying these models, this and the following chapter include also a couple of worked examples in order to get acquainted with the practical applications of models.


% \section{Exponential smoothing}
% Exponential smoothing was developed by Brown \cite{brown1959statistical}, Holt \cite{holt1957forecasting}, Winters \cite{winters1960forecasting}.
% The idea is to forecast future values by taking a weighted average of past observations, where weights decay exponentially in time.
% \begin{equation}
%     \begin{aligned}
%         \hat{y}_{t+h|t}=& l_{t}+hb_{t}+s_{t+h-m(k+1)}
%         \\
%         l_t=&\alpha (y_t-s_{t-m})+(1-\alpha)(l_{t-1}+b_{t-1})
%         \\
%         b_t=& \beta(l_t-l_{t-1})+(1-\beta)b_{t-1}
%         \\
%         s_t=&\gamma(y_t-l_{t-1}-b_{t-1})+(1-\gamma)s_{t-m}
%     \end{aligned}
% \end{equation}
% where $0\leq \alpha \leq 1$ is the smoothing parameter and k is the integer part of $\frac{(h-1)}{m}$.%that is k encodes seasonalities
% Optimal parameters are obtained by minimising the sum of squared errors.


\section{Multiple linear regression}

\section{Autoregressive models}
explain their theory
and how the procedure how they are used
IMPORTANT: load series is not a stationary series so before applying AR we have to perform stationary tests or differencing steps
\subsection{ARIMA}
\subsection{ARIMAX}
\subsection{SARIMA}
\subsection{SARIMAX}

\section{Generalized additive model}
%prophet meta
\section{K-nearest neighbors}

\section{Support vector regression}
Developed at the AT\&T Bell Laboratories by Vapnik et al. \cite{cortes1995support}, \cite{vapnik1997support}, support vector machines SVMs are one of the most popular techniques within the field of statistical learning.
\\
The goal of support vector regression is finding a function f(x) with at most $\epsilon$ deviation from the actual observed data $y_i$ $\forall i$ and as flat as possible. 
Put differently, we would like a model to keep error less than the $\epsilon$ threshold,  
%https://stats.stackexchange.com/questions/5945/understanding-svm-regression-objective-function-and-flatness
In standard SVR we have
\begin{equation}
    f(x)=\langle w,x \rangle +b \ \textrm{with} \ w \in X, b \in R
\end{equation}
Where $X$ denotes the space of the input patters $x_i$.
We can translate the flatness requirement into minimising the squared norm of w. Doing so we can formulate our problem as a convex optimisation problem.
\begin{equation}
    \begin{aligned}
        \min \quad& \frac{1}{2}\|w\|^2
        \\
        s.t. \quad& y_i-\langle w, x_i\rangle-b\leq \epsilon
        \\
        \quad& \langle w, x_i\rangle +b-y_i\leq \epsilon
    \end{aligned}
\end{equation}
Next, we need to introduce the slack variables $\xi$ and $\xi^*$, in order to handle the above formulate optimisation problem.
We obtain the following formulation
\begin{equation}
    \begin{aligned}
        \min \quad& \frac{1}{2}\|w\|^2+C\sum\limits_{i=1}^l(\xi_i+\xi_i^*)
        \\
        s.t. \quad& y_i-\langle w, x_i\rangle-b\leq \epsilon+\xi
        \\
        \quad& \langle w, x_i\rangle +b-y_i\leq \epsilon+\xi^*
        \\
        \quad& \xi_i\geq0
        \\
        \quad& \xi_i^*\geq0
    \end{aligned}
\end{equation}
The C constant trades off between $\epsilon$ deviation tolerance and flatness of the function f.
We seek to minimise the epsilon insensitive loss function, defined as follows
\begin{equation}
    \|\xi\|_\epsilon:=\begin{cases}
        0 \quad& \textrm{if} \ \|\xi\|\leq \epsilon
        \\
        \|\xi\|-\epsilon \quad& \textrm{if} \ \|\xi\|\leq \epsilon
    \end{cases}
\end{equation}
The model is depicted in figure \ref{fig:svm_simple}. The gry band is called the epsilon insensitive tube, only the points outside it are accounted by the loss function.
\begin{figure}
    \includegraphics[width=\textwidth]{images/svm_simple.png}    
    \caption{Support vector regression,\cite{learning_with_kernels}}
    \label{fig:svm_simple}
\end{figure}
Smola et al. \cite{smola2004tutorial} point out that considering the dual formulation makes our optimisation problem easier to solve. Doing so we have
\begin{equation}
    \begin{aligned}
        \max \quad& -\frac{1}{2}\sum\limits_{i,j=1}^l(\alpha_i-\alpha_i^*)(\alpha_j-\alpha_j^*)\langle x_i,x_j\rangle +\sum\limits_{i=1}^l(\alpha_i-\alpha_i^*)(y_i-\epsilon)
        \\
        s.t. \quad& \sum\limits_i=1^l(\alpha_i-\alpha_i^*)=0
        \\
        \quad& \alpha_i, \alpha_i^* \in [0, C]
    \end{aligned}
\end{equation}
Rearranging the gradient of the lagrangian with respect to w, we obtain the so called support vector expansion.
\begin{equation}\label{eq:support vector expansion}
    w=\sum\limits_{i=1}^l(\alpha_i-\alpha_i^*)x_i
\end{equation}
That means that w can be completely described as a linear combination of the training features $x_i$.
Notice, the complexity of the function representation is independent of the feature space dimensionaly but depends only on the number of support vectors; that is those point $i$ for which $(\alpha_i-\alpha_i^*)\neq 0$
Notice that, we do not needd to compute w explicitly in order to evaluate f(x).
Furthermore, equation \ref{eq:support vector expansion} implies
\begin{equation}
    f(x)=\sum\limits_{i=1}^l(\alpha_i-\alpha_i^*)\langle x_i, x\rangle +b
\end{equation}
Employ the Karush-Kuhn-Tucker conditions, b can be retrieved easily. Particularly, use the condition that the product between the constraints and the dual variable has to vanish. These implies that the lagrange multipliers $\alpha_i, \alpha_i^*$ may be nonzero only for the samples inside the epsilon insensitive tube. Consequently for any of these data points, the equality $b=y_i-\langle w, x_i-\epsilon\rangle$ holds.
To get an idea, see figure \ref{fig:svr1} for a how support vector regression handles a sinusoidal function with noise.
\\
\begin{figure}
    \includegraphics[width=\textwidth]{images/svr1.png}
    \caption{support vector regression}
    \label{fig:svr1}
\end{figure}

\section{Artificial neural networks}
\subsection{DNN}
\subsection{LSTM}
\subsection{DeepAr}
\subsection{AR Net}

\section{Kernel methods}
\subsection{Kernel regression}
\subsection{Kernel support vector regression}
Support vector regression can be kernelized by swapping the R2 euclidean dot product of data x with the dot product in the higher feature space F. Doing so, the optimisation problem can be restated as
\begin{equation}
    \begin{aligned}
        \max \quad& -\frac{1}{2}\sum\limits_{i,j=1}^l(\alpha_i-\alpha_i^*)(\alpha_i-\alpha_i^*)k(x_i, x_j)+\sum\limits_{i=1}^l(y_i-\epsilon)(\alpha_i-\alpha_i^*)
        \\
        s.t. \quad& \sum\limits_{i=1}^l(\alpha_i-\alpha_i^*)=0
        \\
        \quad& \alpha_i, \alpha_i^* \in [0,C]
    \end{aligned}
\end{equation}
Our regressor f is then given by
\begin{equation}
    \begin{aligned}
        \sum\limits_{i=1}^l(\alpha_i-\alpha_i^*)k(x_i, x)+b
    \end{aligned}
\end{equation}
Notice, in this setting w is no longer given explicitly and we seek for the flattest function in the feature space not the input space.
See figure \ref{fig:svr2} and figure \ref{fig:svr3} for two examples. In the former we have used a polynomial kernel while in the latter the standard radial basis function kernel.
\begin{figure}
    \includegraphics[width=\textwidth]{images/svr2.png}
    \caption{polynomial support vector regression}
    \label{fig:svr2}
\end{figure}

\begin{figure}
    \includegraphics[width=\textwidth]{images/svr3.png}
    \caption{rbf support vector regression}
    \label{fig:svr3}
\end{figure}
Comparing theese pictures with figure \ref{fig:svr1}, it can be concluded that introducing kernels allows support vector regression to handle non linearities in the data.

\chapter{Probabilistic forecasting}\label{ch:prob}
When it comes to probabilistic forecasting, there are a couple of standard practical approaches; conceptually they can be grouped into two main categories. The former class of approaches tries modelling the distribution of the observed data directly. The latter family of approaches constructs first a point forecast and then learns the distribution of the model's errors.
\section{Quantile regression}
- Explain the theory of quantile regression

\section{Kernel density estimation}
- Explain the framework of kernel density estimation.

- Explain how it is applied in the literature.

- Extend to Conditional kernel density estimation
and how it is applied in the literature

- there is sample in sklearn kde, so we can use it to create a method of the kde class that computes its crps.

- Do a simple showcase with an example

\section{Ensemble methods}
- Explain the idea of ensemble methods.

- The most popular framework is based on
autoregressive processes

- simple example

% \section{Copula models}
\section{Quantile forest}
Meinshausen \cite{meinshausen2006quantile} extends the idea of random forest \cite{breiman2001random} generalising it, the result is the quantile forest algorithm. Quantile forest allows us to estimate conditional quantiles in a non parametric fashion.
\\
In order to understand this algorithm it is necessary to first cover the theory of decision trees and random forests.
- decision trees
- random forest
- quantile forest
\section{Quantile gradient boosting machine}


\section{DMLP}
%gluon nn
% train a beta distribution to learn its parameters
\section{DeepAR}

\section{Kernel methods}
\subsection{Kernel quantile regression}
The idea of quantile regression has been extended to kernel methods by Takeuchi et al. \cite{takeuchi2006nonparametric}.
There, they minimize a risk plus regularizer defined as follows.
\begin{equation}\label{eq:kqr_min1}
    R[f]:=\frac{1}{m}\sum\limits_{i=1}^{m}\rho_q(y_i-f(x_i))+\frac{\lambda}{2}\|g\|_H^2
\end{equation}
where $f=g+b$, $g \in H$ and $b \in R$.
Using the link between RKHS and feature spaces, we can rewrite $f(x)=\langle w, \phi(x) \rangle+b$. Moreover, note that the RKHS norm is defined as follows $\|f\|_{H}=\inf\{\|w\|_{F}:w\in F,f(x)=\langle w,\varphi (x)\rangle _{F},\forall x\in X\}$, where $F$ is the feature space.
 Doing so we obtain a minimization problem equivalent to minimizing equation \ref{eq:kqr_min1}.
\begin{equation}\label{eq:kqr_min2}
    \begin{aligned}
    \min_{w,b} \quad & C \sum \limits_{i=1}^{m}
    q(y_i-\langle w, \phi(x) \rangle-b)+ (1-q)(-y_i+\langle w, \phi(x) \rangle+b)+ \frac{1}{2}\|w\|^2\\
    \end{aligned}
    \end{equation}
Note the division by $\lambda$ so that $C=\frac{1}{\lambda m}$.
\\
We can next rephrase the optimisation in \ref{eq:kqr_min2} by introducing the slack variables $\xi_i$ and $\xi_i^*$.
\begin{equation}\label{eq:kqr_min3}
    \begin{aligned}
        \min_{w,b,\xi_i,\xi_i^*} \quad & C \sum \limits_{i=1}^{m}
        q \xi_i+ (1-q)\xi_i^*+ \frac{1}{2}\|w\|^2\\
    \textrm{s.t.} \quad & y_i-\langle w, \phi(x) \rangle-b \leq \xi_i\\
    & -y_i+\langle w, \phi(x) \rangle+b \leq \xi_i^*\\
      &\xi_i\geq0    \\
      &\xi_i^*\geq0    \\
    \end{aligned}
    \end{equation}
In order to make it more compact, we rewrite equation \ref{eq:kqr_min3} in matrix notation.
\begin{equation}\label{eq:kqr_min4}
    \begin{aligned}
        \min_{w,b\xi,\xi^*} \quad & C q \xi^\intercal \mathbb{1}+ C q (\xi^*)^\intercal \mathbb{1}+ \frac{1}{2}w^\intercal w\\
    \textrm{s.t.} \quad & y-\Phi^\intercal w -b \preceq \xi\\
    & -y+\Phi^\intercal w +b \preceq \xi^*\\
      &\xi\succeq0    \\
      &\xi^*\succeq0    \\
    \end{aligned}
    \end{equation}
Consider now, the lagrangian $L$ associated to \ref{eq:kqr_min4}
\begin{equation}\label{eq:kqr_min5}
    \begin{aligned}
    L(w,b,\xi,\xi^*)= & C q \xi^\intercal \mathbb{1}+ C q (\xi^*)^\intercal \mathbb{1}+ \frac{1}{2}w^\intercal w- \alpha^\intercal(\xi - y+\Phi^\intercal w +b)
    - (\alpha^*)^\intercal(\xi^* +y-\Phi^\intercal w -b)
    \\
    & -\nu^\intercal \xi - (\nu^*)^\intercal \xi^*
\end{aligned}
\end{equation}
The next step is deriving its dual formulation, since it is easier and more efficient to solve. This because the dual problem has the useful property of being always convex.

\begin{definition}
    The dual function associated to the lagrangia $L(x,\lambda)$ is given by $g(v)=\inf_x L(x,\lambda)$
\end{definition}
where $\lambda$ is called the lagrange multiplier of the optimization problem. Such dual formulation has an useful property, that is \begin{equation}\label{weak_duality}
    g(\lambda)\leq p^*
\end{equation}
where $p^*$ is the optimal value of your optimization problem.
Consider a simple lagrangian $L(x,\lambda)=f(x)+\sum \lambda_i r_i(x) +\sum v_i h_i(x)$, where $r_i(x)$ are inequality constraints while $h_i(x)$ are equality constraints of the problem. Then it can be noted that the lower bound on $p^*$ is non trivial only when the lagrange multiplier $\lambda \succeq 0$. Therefore, the idea is that by maximizing the dual function subject to the contraint $\lambda \succeq 0$, we can obtain an approximate or perfect solution to the primal problem.
\\
To explain why we may or not be able to attain the best solution to the primal problem by maximizing its dual we have to introduce the concept of duality.
\\
We use $d^*$ to denote the optimal value of the lagrange dual problem; we can think of it as the best lower bound on $p^*$. 
\\
The inequality \ref{weak_duality} is called weak duality. The difference $p^*-d^*$ is said the optimal duality gap; it is the gap between the optimal value of the primal problem and the best lower bound on it that can be obtained from the Lagrange dual function. Moreover, note that the optimal duality gap is always nonnegative.
\\
We say that strong duality holds, when the optimal duality gap is zero; in other words, the lagrange dual bound is tight.
\\
Constraint qualifications are condtions under which strong duality holds; one of the most popular is Slater's condition.
\begin{equation}\label{slater_condition}
    \begin{aligned}
        \exists x \in \textrm{relint} \ D \ \textrm{s.t.} & \ r_i(x)<0, \quad i=1, \dots, m \\
        & h_i(x)=b
    \end{aligned}
\end{equation}
Where relint $D$ is the relative interior of $ D:=\cap _{i=0}^{m}\operatorname {dom} (r_{i})$
\\
Slater's theorem naturally follows.
\begin{theorem}
    If Slater's condtion holds and the problem is convex then strong duality holds.
\end{theorem}
We can now check that our optimization problem posseses strong duality by checking Slater's condition.
\\
In our case we don't have any equality constraint, so we do not have to worry about the $h_i(x)=0$ term in \ref{slater_condition}. All we have to check is the convexity of our problem and that there exist a $x$ such that $r_i(x)<0$.
For convexity, a sufficient condition is the positive definiteness of $Q$ in the quadratic programming problem 
\begin{equation}
    \begin{aligned}
        \min \quad & x^\intercal Q x+ c^\intercal x \\        
        s.t \quad& Ax\preceq b
    \end{aligned}
\end{equation}
In our case \ref{eq:kqr_min5}, we have $w^\intercal w$, thus $Q$ is just the identity matrix which satisfies the positive definiteness requirement. Therefore, our problem is convex.
Next we check that Slater's condition holds. Considering first the two non negative constraints on $\xi$ and $\xi^*$, we conclude that $\xi$  and $\xi^*$ have to be greater or equal to zero for the existence of an $x$ satisfying Slater's condition. Thus, let us suppose that $0 \leq \xi \leq \alpha$ and $0 \leq \xi^* \leq \alpha$.
\\
Next, let us consider the other two inequalities and make the following ansatz.
\\
\begin{equation}
    \begin{aligned}
        w=& \Phi^\intercal(\Phi \Phi^\intercal)^{-1} y\\
        b<& \alpha
    \end{aligned}
\end{equation}
We then have
\begin{equation}
    \begin{aligned}
        -\xi + y -\Phi\Phi^\intercal(\Phi \Phi^\intercal)^{-1}y-b<&0
        \\
        -\xi^* - y +\Phi\Phi^\intercal(\Phi \Phi^\intercal)^{-1}y+b<&0
    \end{aligned}
\end{equation}
Hence, we conclude that our problem satisfies Slater's condition. Therefore the solution of the dual and primal problem are equivalent.
\\
We end this section with the derivation of the dual problem; that is the convex problem we will solve in order to get the qunatiles prdeiction of our quantile kernel algorithm.
\\
First, derive the dual function of \ref{eq:kqr_min5}.
\begin{equation}
    \begin{aligned}
        g(\alpha, \alpha^*, \nu, \nu^*)=& \inf_x L(x,\lambda)\\
    = & \inf_{\xi, \xi^*, w, b} L(w,b,\xi,\xi^*)
\end{aligned}
\end{equation}
Setting its derivates to zero
\begin{equation}\label{eq:lagrange_derivatives}
    \begin{cases}
        \frac{\partial L}{\partial w}=0 \implies w=\Phi^\intercal(\alpha-\alpha^*)
        \\
        \frac{\partial L}{\partial b}=0 \implies \alpha-\alpha^*=0
        \\
        \frac{\partial L}{\partial \xi}=0 \implies Cq \mathbb{1}-\alpha- \nu=0
        \\
        \frac{\partial L}{\partial \xi^*}=0 \implies C(1-q)\mathbb{1} -\alpha^* -\nu^*=0
    \end{cases}
\end{equation}
As pointed out previously, the lower bound resulting from the dual formulation is non trivial only when the lagrange multipliers are $\succeq 0$. Looking at the last two equations of the system \ref{eq:lagrange_derivatives}, this implies the following two constraints $\alpha \in [0, Cq\mathbb{1}]$ and $\alpha \in [0, C(1-q)\mathbb{1}]$.
\\
Substitute the conditions for an optimum into \ref{eq:kqr_min5} we obtain the dual formulation.
\begin{equation}
    \begin{aligned}
        g(\alpha, \alpha^*)=& \xi^\intercal(Cq\mathbb{1}-\alpha -\nu)+(\xi^*)^\intercal(C(1-q)\mathbb{1}-\alpha^*-\nu^*)-(\alpha-\alpha^*)^\intercal \Phi\Phi^\intercal(\alpha-\alpha^*)
        \\
        & +(\alpha-\alpha^*)^\intercal y-(\alpha-\alpha^*)^\intercal b+\frac{1}{2}(\alpha-\alpha^*)^\intercal \Phi\Phi^\intercal(\alpha-\alpha^*)
        \\
        \\
        g(\alpha, \alpha^*)=& 0+0-\frac{1}{2}(\alpha-\alpha^*)^\intercal \Phi\Phi^\intercal(\alpha-\alpha^*)+(\alpha-\alpha^*)^\intercal y-0
        \\
        g(\alpha, \alpha^*)=& -\frac{1}{2}(\alpha-\alpha^*)^\intercal \Phi\Phi^\intercal(\alpha-\alpha^*)+(\alpha-\alpha^*)^\intercal y
    \end{aligned}
\end{equation}
Defining $\alpha=(\alpha-\alpha^*)$ and letting $K$ the kernel matrix, we have that the dual optimisation problem reads as follows
\begin{equation}\label{eq:kqr_min6}
    \begin{aligned}
        \max_{\alpha} \quad & -\frac{1}{2}\alpha^\intercal K\alpha+\alpha^\intercal y\\
    \textrm{s.t.} \quad & 
    C(q-1)\mathbb{1}\preceq \alpha \preceq Cq\mathbb{1}\\
    &\alpha^\intercal\mathbb{1}=1
    \end{aligned}
    \end{equation}
Switching sign, we rephrase it as a minisation problem, which is the common practice in convex optimisation.
\begin{equation}\label{eq:kqr_min7}
    \begin{aligned}
        \min_{\alpha} \quad & +\frac{1}{2}\alpha^\intercal K\alpha-\alpha^\intercal y\\
    \textrm{s.t.} \quad & 
    C(q-1)\mathbb{1}\preceq \alpha \preceq Cq\mathbb{1}\\
    &\alpha^\intercal\mathbb{1}=1
    \end{aligned}
    \end{equation}
The kernel quantile regression estimator is then given by
\begin{equation}
    f(x)=\sum\limits_i \alpha_i k(x_i, x)+b
\end{equation}
Where an unbiased estimator for $b$ is given by
\begin{equation}
    \begin{aligned}
    b=&\frac{1}{I_q}\sum\limits_{i\in I_q} b_i, \\ 
    \textrm{with} \ b_i=& y_i-\sum\limits_k \alpha_k k(x_k, x_i) \\
     \textrm{and} \ I_q=& \{i=1,\dots,n|0<\alpha_i<Cq, 0<\alpha_i<C(1-q)\}
    \end{aligned}
\end{equation}
%contribution is implementation kernel quantile regression for pythos users since there exists only r version
\subsection{Kernel herding}
Select the best point forecasting method and create a probabilistic forecast by modelling its model errors with kernel herding (do something like the residual bootstrap ensembles if it is meaningful and possible to implement).

\chapter{Exploratory analysis and data extract transform load pipeline}\label{eda}
- Explain how data has been retrieved.

- Data provider EPEX(entsoe retrieves its data)

- Explain the ETL(Extract Transform Load) pipeline
I set up.

- Explore data in order to get useful insights for
how to tune the models to get the most of them.

- Correlation and auto correlation plots

- Split train and test dataset
Explain carefully why it is important to carry
out an out of sample test and not in sample.
In sample test involves look ahead bias because we are
fitting the model on the data we want to predict,
thus it overfits on the data considered but it does
not generalize well.

\chapter{Implementation}\label{implementation}
This section is intended to explain and aid for reproducibility studies. Hereafter, the specific libraries used and the custom implementatios are thoroughly documented.
\\
For the list of Python packages needed see the requirement.txt file on \url{https://github.com/luca-pernigo/ThesisKernelMethods}\label{github_repo}.
% - indicate computer specifics
All experiments have been carried out on a 3.2 GHz 16GB Apple M1 Pro.


% Section documenting code

% - Explain how methods' implementation has been
% adapted to my specific setting.

% - Explain in detail how to my src code has been implemented
% its rationale and how to use it.

% - As I explain code scripts go over the test, to 
% explain better my ideas.

% - Indicate also hyperparameters maybe in each subsection

\section{Point forecasting}
\subsection{Multiple linear regression}
For multiple linear regression the one from the \href{https://scikit-learn.org/stable/}{sklearn} library has been used.

\subsection{Trigonometric seasonality Box-Cox transformation ARMA errors trend and seasonal components}
Tbats implementation is available at \url{https://github.com/intive-DataScience/tbatsx}.
In our application we specified as hyperparameters the length of seasons; 24 for the daily seasonality and 168 for the weekly seasonality.

\subsection{Prophet}
The \href{https://facebook.github.io/prophet/docs/quick_start.html}{prophet} model has been applied by employing the Python API provided by Meta.

\subsection{K-nearest neighbours}
The object KNeighborsRegressor of the \href{https://scikit-learn.org/stable/}{sklearn} module neighbors has been used.

\subsection{Support vector regression}
The object SVR of the \href{https://scikit-learn.org/stable/}{sklearn} module svm has been used by specifying the linear kernel.

\subsection{Long short term memory}
The LSTM predictor has been built using the \href{https://pytorch.org}{torch} library.

\subsection{Kernel ridge regression}
The object KernelRidge of the \href{https://scikit-learn.org/stable/}{sklearn} module kernel\_ridge has been used.

\subsection{Kernel support vector regression}
The object SVR of the \href{https://scikit-learn.org/stable/}{sklearn} module svm has been used.
% by specifying rbf as the kernel parameter.

\section{Probabilistic forecasting}
\subsection{Linear quantile regression}
The implementation of \href{https://www.statsmodels.org/dev/generated/statsmodels.regression.quantile_regression.QuantReg.html}{quantile\_regression.QuantReg} from the regression module of statsmodels has been used.
The model is fitted through iterative reweighted least squares.
\subsection{Quantile gradient boosting machine}
The implementation of \href{https://scikit-learn.org/stable/modules/generated/sklearn.ensemble.GradientBoostingRegressor.html}{GradientBoostingRegressor} from the sklearn.ensemble submodule has been used.
\subsection{Quantile forest}
The implementation of \href{https://pypi.org/project/quantile-forest/}{quantile\_forest.RandomForestQuantileRegressor} has been used. This estimator is compatible with scikit-learn API \cite{Johnson2024}.
\subsection{Kernel quantile regression}
Kernel quantile regression had no previously implemented Python open source library, thus the need of implementing our own version.
\\
The scikit-learn team provides a project template for the creation of estimators compatible with scikit-learn functionalities. Therefore, the KQR class is derived from the scikit-learn BaseEstimator and the mixin class RegressorMixin.
Our KQR class is initialized by providing a quantile, the regularization term C, the kernel family and its corresponding hyperparameters.
\\
In the fit method, we set up and solve the convex optimisation problem through the interior point method. This algorithm is taken from the cvxopt library, see its official manual \citeW{vandenberghe2010cvxopt} for a reference.
When using this library, it is important to keep two things in mind. First this library assumes the quadratic term of the optimisation problem to be multiplied by the 0.5 factor, thus we just have to provide the $Q$ matrix with no 0.5 in front.
Secondly, in order to specify multiple inequalities we have to stack them and provide them as a unique matrix.
\\
Once a solution to the convex problem has been found, we create a mask for the support vectors of the estimator in order to estimate the constant term of our kernel quantile regressor.
\\
In the predict method, we pass a matrix $X\_eval$ of independent variables, next we compute the kernel matrix between $X\_train$ and $X\_eval$ and obtain $y\_eval$ with the formula $y=\alpha^\intercal K+b$.
\\
This estimator is compatible with useful scikit-learn in built methods like gridsearch, crossvalidation and scoring rules. Moreover, this code is compatible with sklearn in built kernel functions \href{https://scikit-learn.org/stable/modules/classes.html#module-sklearn.metrics.pairwise}{sklearn.metrics.pairwise} and \href{https://scikit-learn.org/stable/modules/classes.html#module-sklearn.gaussian_process.kernels}{sklearn.gaussian\_process.kernels}, functions supported by our kernel quantile regression include: Gaussian RBF, absolute Laplacian, Matern 3/2, Matern 5/2, Linear, Cosine, Sigmoid, Periodic, Polynomial and custom composition of kernels.
% NOTICE, these kernel implementations differ in the form of passed parameters in the former we pass in the numerator gamma while in the latter we pass l, which stands for length scale in the denominator. To cross validate overt the same range of parameter across the different kernels we made sura that gamma=1/l.


% \\
% Up to now, there are only two open source implementation of the quantile kernel regression. Nevertheless they are both in R, that is there exists no python, matlab or julia open source implementation. 

% Following are reported the results of a comparative study between our own implementation and the one of the R library kernlab


% (i guess it is in C or C++ and then binded to R).
% \subsubsection{Python versus R implementation}
% In this section, a comparison study has been carried out in order to inspect the competitiveness of our implementation with the existing one for the R programming language.

\chapter{Experiments analysis}\label{analysis}
Analysis of experiments and results
\\
Building on the theory introduced in \ref{ch:point} and \ref{ch:prob}, this section covers the experiments carried out and the results obtained.

% - Comments
% \\
% - Comparison
% \\
% - Table of loss scores
% \\
% Plots:
% \\
% - Plots for visualizing timeseries with quantiles bounds
% \\
% - Other plots that will come up to mind

\section{Point forecasting}
This section carries out a comparative study between the state of the art methods for point forecasting, introduced in \ref{ch:point}.
As use case, we will consider the task of predicting the electric load from the GEFCom 2014 dataset.
In such setting we considered the following regressors
\begin{itemize}
    \item Day
    \item Hour
    \item Month
    \item Day of the week
    \item Is holiday
    \item Weather temperature
\end{itemize}
Methods will be compared by means of the RMSE, MAE and MAPE scores, see section \ref{metrics}.
\subsection{Multiple linear regression}
To get started, standard multiple linear regression has been applied, see fig \ref{fig:mlr_price} for a visualisation. 
\begin{figure}
    \includegraphics[width=\textwidth]{images/mlr_price.png}
    \caption{Multiple linear regression prediction}
    \label{fig:mlr_price}
\end{figure}

The resulting RMSE is 30.59.
What can be concluded, is that multiple linear regression is capable of catching the daily seasonality. Nevertheless, it cannot catch the range of the price series properly.

\subsection{Trigonometric seasonality Box-Cox transformation ARMA errors trend and seasonal components}
Next, we tried with the autoregressive approach. Unfortunately, AR, ARIMA and SARIMA models did not perform as expected, their output was slightly better than the one of linear regression. This is probably due to the fact that, the data considered entails two kinds of seasonalities, while ARIMA models can only handle one at a time. We remind the reader, that the electricity load time series involves both daily and weekly seasonalities. Hence, the need for a more advanced time series model. The Tbats \cite{de2011forecasting} model is a forecasting method capable of handling complex patterns in the data. Its name stands for trigonometric seasonality, Box-Cox transformation, ARMA errors, trend and seasonal components. 
Tbats forecast is visualized in figure \ref{fig:tbats_price}, meanwhile its RMSE is 15.08.
\begin{figure}[!h]
    \includegraphics[width=\textwidth]{images/tbats_price.png}
    \caption{Tbats prediction}
    \label{fig:tbats_price}
\end{figure}
From the plot we see that Tbats is capable of catching the lows and average trend, conversely it has some difficulties handling the price peaks.
\subsection{Prophet}
Following, the prophet model has been considered. 
We get started by considering the base implementation. In such setting, prophet takes in as input the time series object and learns its data generating process.
This method achieves a RMSE of 23.96, its prediction is visualized in figure \ref{fig:prophet_price_1}.
\begin{figure}[!h]
    \includegraphics[width=\textwidth]{images/prophet_price_1.png}
    \caption{Prophet prediction}
    \label{fig:prophet_price_1}
\end{figure}
What can be seen is that, prophet models correctly the average trend but does not model peaks and lows precisely. 
Next, a more complex model was trained. We added the weather temperature, the square of it and the categorical variable for holidays effect as regressors. Furthermore, we also applied a log transformation to the dependet variable. Doing so, RMSE went down to 10.29. Forecast is visualized in \ref{fig:prophet_price2}, moreover, figure \ref{fig:prophet_price2.1} and figure \ref{fig:prophet_price2.2} break down the trend and the seasonalities respectively.
\begin{figure}[!h]
    \includegraphics[width=\textwidth]{images/prophet_price2.png}
    \caption{Prophet v2 prediction}
    \label{fig:prophet_price2}
\end{figure}

\begin{figure}[!h]
    \includegraphics[width=\textwidth]{images/prophet_price2.1.png}
    \caption{Seasonalities breakdown}
    \label{fig:prophet_price2.1}
\end{figure}

\begin{figure}[!h]
    \includegraphics[width=\textwidth]{images/prophet_price2.2.png}
    \caption{Prices trend}
    \label{fig:prophet_price2.2}
\end{figure}
Prophet alongside with the right features is a good model in the context of electricity price forecasting, it is capcable of catching the trend, the peak and the lows. 

\subsection{K-nearest neighbours}
Afterwards, K-nearest neighbours has been considered. In doing so, we cross validated the number of neighbours to get the best model instance. The forecast is visualized in figure \ref{fig:knn_price}, RMSE is 12.54.
\begin{figure}[!h]
    \includegraphics[width=\textwidth]{images/knn_price.png}
    \caption{K-nearest neighbour regression}
    \label{fig:knn_price}
\end{figure}
What it can be said is that, K-nearest neighbour is capable of predicting prices well by averaging past data.

\subsection{Support vector regression}
Coming up we have support vector regression. In applying this model we used gridsearch crossvalidation to search for the best regularization parameter C. Support vector regression achieves an RMSE of 23.24, for the prediction see figure \ref{fig:svr_price}.
\begin{figure}[!h]
    \includegraphics[width=\textwidth]{images/svr_price.png}
    \caption{Support vector regression prediction}
    \label{fig:svr_price}
\end{figure}
Visually, we see that the SVR performs similar to multiple linear regression, as expected. Like multiple linear regression, SVR can model the daily seasonality but the point prediction is off in terms of highs and lows.


\subsection{Long short term memory}
We used the torch library in order to code our LSTM regressor.
Various hyperparameters combinations were tried during hyperparameter tuning. The final hyperparameters we set on follows.
\begin{itemize}
    \item hidden\_size= 64
    \item num\_layers= 2
    \item output\_size= 1
    \item num\_epochs= 30
    \item learning\_rate= 0.001
    \item batch\_size= 32
    \item window\_size= 24
\end{itemize}
LSTM forecast is reported in table \ref{fig:lstm_price}, it achieves a RMSE of 9.1782
\begin{figure}[!h]
    \includegraphics[width=\textwidth]{images/lstm_price.png}
    \caption{Long short term memory prediction}
    \label{fig:lstm_price}
\end{figure}

\subsection{Kernel ridge regression}
Subsequently, kernel ridge regression was considered. The kernel considered is the radial basis gaussian function.
Cross validation was carried jointly over the rbf kernel bandwith and the regularization constant.
The RMSE achieved is 9.86, figure \ref{fig:krnridge_price} reports the prediction.
\begin{figure}[!h]
    \includegraphics[width=\textwidth]{images/krnridge_price.png}
    \caption{Kernel ridge prediction}
    \label{fig:krnridge_price}
\end{figure}
We can observe that, kernel ridge regression models accurately the electricity time series.

\subsection{Kernel support vector regression}
The last model we considered in this setting was the kernel support vector regression.
RMSE achieved with this method is 9.10, forecast is reported in figure \ref{fig:krnsvr_price}.

\begin{figure}[!h]
    \includegraphics[width=\textwidth]{images/krnsvr_price.png}
    \caption{Kernel support vector prediction}
    \label{fig:krnsvr_price}
\end{figure}
We can observe that kernel support vector regression is one of the best performing techniques between the one considered.

\subsection{Results}
This section reports the table comparing the considered methods scores 
% in the context of electricity load forecasting.

% - table 
% |methods|tasks RMSE|
% |       |X          |       
% |       |X          |    
Table \ref{tab:point_RMSE} reporst the RMSE scores for each of the considered methods. It can be seen that kernel based methods are the ones achieving the lowest RMSE; specifically, we have KrnSVR, KrnRidge and KNN being among the top methods on the majority of tasks.

\begin{table}[!ht]
    \caption{Root mean squared errors}
    \label{tab:point_RMSE}
    \begin{adjustbox}{width=\textwidth, height=3cm}
        \begin{tabular}{lrrrrrrrrrrrrrrr}
            \toprule
            Method/RMSE & Task 1 & Task 2 & Task 3 & Task 4 & Task 5 & Task 6 & Task 7 & Task 8 & Task 9 & Task 10 & Task 11 & Task 12 & Task 13 & Task 14 & Task 15 \\
            \midrule
            MLR & 30.5892 & 29.0347 & 75.1682 & 63.1376 & 42.1156 & 36.6233 & 39.2496 & 38.8286 & 47.1307 & 68.0494 & 61.6005 & 30.3751 & 34.5523 & 33.4287 & 33.5581 \\
            TBATS & 15.0886 & 31.9720 & 99.1260 & 84.4890 & 51.9740 & 34.9055 & 18.4445 & 38.3246 & 74.1117 & 98.6600 & 84.3050 & 38.8433 & 16.6080 & 28.9902 & 41.6194 \\
            Prophet & 10.2936 & 14.4358 & 38.7551 & 63.4787 & 19.7474 & 17.5065 & 12.6926 & 14.2665 & 17.5466 & 23.5944 & 43.6666 & 20.8637 & 16.9493 & 19.1626 & 23.3889 \\
            KNN & 12.5429 & 11.6699 & 24.5057 & 18.3310 & 13.1821 & \textbf{11.9238} & 12.0044 & 14.5165 & 16.3132 & 15.1831 & 37.6457 & 16.4690 & 12.0324 & 11.3102 & 14.1717 \\
            SVR & 23.2421 & 25.8525 & 78.8043 & 67.6209 & 42.3748 & 32.7845 & 30.5971 & 35.3660 & 55.4213 & 77.9660 & 68.5799 & 30.2451 & 27.0345 & 28.3652 & 32.1480 \\
            LSTM & 9.1782 & 12.0669 & \textbf{22.7048} & \textbf{16.2087} & 16.0964 & 12.7936 & \textbf{10.8559} & 14.6173 & 19.7303 & 18.0200 & 43.2051 & 17.1856 & 10.3106 & 12.1347 & 17.5849 \\
            KrnRidge & 9.8619 & 11.6101 & 37.7824 & 35.2459 & \textbf{12.5160} & 14.5911 & 12.8791 & 17.4385 & 16.1131 & 17.1938 & 37.6961 & 14.0076 & 9.8441 & 10.7491 & 13.2975 \\
            KrnSVR & \textbf{9.1028} & \textbf{11.3117} & 22.9281 & 19.2132 & 12.6331 & 12.6018 & 11.3537 & \textbf{12.9506} & \textbf{14.9731} & \textbf{11.7765} & \textbf{37.3797} & \textbf{13.2694} & 
            \textbf{8.8522} & \textbf{10.6185} & \textbf{13.2602} \\
            \bottomrule
            \end{tabular}            
    \end{adjustbox}
\end{table}


The MAE scores are contained in table \ref{tab:point_MAE}. Similarly to above, we can conclude that kernel methods stand out for achieving also the lowest MAE score.
\begin{table}[!ht]
    \caption{Mean absolute errors}
    \label{tab:point_MAE}
    \begin{adjustbox}{width=\textwidth, height=3cm}
        \begin{tabular}{lrrrrrrrrrrrrrrr}
            \toprule
             Method/MAE & Task 1 & Task 2 & Task 3 & Task 4 & Task 5 & Task 6 & Task 7 & Task 8 & Task 9 & Task 10 & Task 11 & Task 12 & Task 13 & Task 14 & Task 15 \\
            \midrule
            MLR & 27.7219 & 24.3823 & 65.4016 & 52.3873 & 34.8465 & 31.2663 & 35.9420 & 34.7316 & 37.0273 & 54.7075 & 48.9327 & 26.1975 & 31.8720 & 29.2306 & 28.3428 \\
            TBATS & 10.5408 & 23.8417 & 91.3153 & 74.5734 & 40.3258 & 26.0092 & 14.2141 & 24.7818 & 62.0452 & 86.7125 & 72.6912 & 28.4235 & 11.1184 & 21.1038 & 31.8555 \\
            Prophet & 8.6139 & 11.4307 & 28.3509 & 46.4480 & 15.6290 & 14.1278 & 10.1574 & 11.2360 & 14.0072 & 18.9565 & 27.1142 & 16.9678 & 14.6615 & 16.1750 & 18.3043 \\
            KNN & 9.4003 & 9.1126 & 19.6213 & 14.4906 & 10.6461 & \textbf{9.4354} & 8.7441 & 10.0065 & 12.4714 & 11.2006 & 20.3087 & 12.5416 & 9.2558 & 8.5937 & 10.5451 \\
            SVR & 20.2443 & 21.2526 & 69.3704 & 57.1107 & 34.3651 & 27.5875 & 27.4651 & 29.4204 & 43.7009 & 64.3522 & 55.4307 & 25.1080 & 24.0294 & 24.4049 & 26.5869 \\
            LSTM & 7.5217 & 9.4767 & \textbf{17.9851} & \textbf{13.2158} & 11.8118 & 9.8607 & \textbf{8.0561} & 10.8480 & 15.1008 & 13.8108 & 25.9906 & 13.3588 & 8.1526 & 9.8000 & 14.3186 \\
            KrnRidge & 7.5965 & 9.2130 & 28.2484 & 24.2366 & \textbf{10.0589} & 10.7758 & 9.5788 & 11.2166 & 12.3427 & 12.3306 & 20.0213 & 10.8434 & 7.1625 & \textbf{7.9710} & 10.0767 \\
            KrnSVR & \textbf{6.9581} & \textbf{8.6989} & 18.7739 & 15.1398 & 10.1395 & 9.5927 & 8.5731 & \textbf{9.0033} & \textbf{11.0618} & \textbf{8.5486} & \textbf{19.7383} & \textbf{10.5545} & \textbf{7.0395} & 8.1926 & \textbf{9.8125} \\
            \bottomrule
            \end{tabular}            
    \end{adjustbox}            
\end{table}

Finally table \ref{tab:point_MAPE} reports the MAPE score for each method. What can be concluded is that kernel regressors are also the best in terms of MAPE score; with KrnSVR being the top one.
\begin{table}[!ht]
    \caption{Mean absolute percentage errors}
    \label{tab:point_MAPE}
    \begin{adjustbox}{width=\textwidth, height=3cm}
        \begin{tabular}{lrrrrrrrrrrrrrrr}
            \toprule
             Method/MAPE & Task 1 & Task 2 & Task 3 & Task 4 & Task 5 & Task 6 & Task 7 & Task 8 & Task 9 & Task 10 & Task 11 & Task 12 & Task 13 & Task 14 & Task 15 \\
            \midrule
            MLR & 0.2470 & 0.1935 & 0.3030 & 0.2601 & 0.2474 & 0.2667 & 0.3498 & 0.3058 & 0.1954 & 0.2432 & 0.4234 & 0.2114 & 0.2926 & 0.2502 & 0.2145 \\
            TBATS & 0.0833 & 0.1650 & 0.4285 & 0.3731 & 0.2482 & 0.1849 & 0.1234 & 0.1634 & 0.3181 & 0.4021 & 0.4880 & 0.1761 & 0.0905 & 0.1555 & 0.2058 \\
            Prophet & 0.0737 & 0.0854 & 0.1324 & 0.2453 & 0.1134 & 0.1169 & 0.0938 & 0.0912 & 0.0880 & 0.0992 & 0.3757 & 0.1389 & 0.1343 & 0.1356 & 0.1370 \\
            KNN & 0.0778 & 0.0682 & 0.0924 & 0.0761 & 0.0751 & 0.0740 & 0.0784 & 0.0744 & 0.0725 & 0.0555 & \textbf{0.3115} & 0.0888 & 0.0816 & 0.0699 & 0.0765 \\
            SVR & 0.1808 & 0.1640 & 0.3211 & 0.2829 & 0.2317 & 0.2260 & 0.2676 & 0.2451 & 0.2215 & 0.2867 & 0.4354 & 0.1870 & 0.2216 & 0.2044 & 0.1933 \\
            LSTM & 0.0642 & 0.0706 & 0.0896 & \textbf{0.0722} & 0.0798 & 0.0772 & \textbf{0.0710} & 0.0854 & 0.0909 & 0.0741 & 0.3578 & 0.0976 & 0.0719 & 0.0778 & 0.1037 \\
            KrnRidge & 0.0625 & 0.0685 & 0.1281 & 0.1207 & 0.0718 & 0.0797 & 0.0812 & 0.0760 & 0.0696 & 0.0587 & 0.3127 & 0.0779 & \textbf{0.0591} & \textbf{0.0637} & 0.0722 \\
            KrnSVR & \textbf{0.0568} & \textbf{0.0642} & \textbf{0.0892} & 0.0788 & \textbf{0.0714} & \textbf{0.0724} & 0.0740 & \textbf{0.0665} & \textbf{0.0629} & \textbf{0.0428} & 0.3122 & \textbf{0.0781} & 0.0607 & 0.0646 & \textbf{0.0713} \\
            \bottomrule
            \end{tabular}            
    \end{adjustbox}
\end{table}

\section{Probabilistic forecasting}
In this subsection we compare performance of quantile regressors.
\\
- experiment for quantile estimator on gefcom2014 and we are good


\section{Conclusions}


%SYMBOLS LISTINGS

%For mathematical terms \ensuremath
% \glsxtrnewsymbol[description={Autoregressive exogenous model}]{ARX}{\ensuremath{ARX}}

%A
\glsxtrnewsymbol[description={Artificial neural network}]{ANN}{ANN}
\glsxtrnewsymbol[description={Autoregressive model}]{AR}{AR}
\glsxtrnewsymbol[description={Autoregressive moving average model}]{ARMA}{ARMA}
\glsxtrnewsymbol[description={Autoregressive integrated moving average model}]{ARIMA}{ARIMA}
\glsxtrnewsymbol[description={Autoregressive exogenous model}]{ARX}{ARX}

%C
\glsxtrnewsymbol[description={Computational intelligence}]{CI}{CI}
\glsxtrnewsymbol[description={Conditional kernel density}]{CKD}{CKD}
\glsxtrnewsymbol[description={Continous ranked probability score}]{CRPS}{CRPS}

%D
\glsxtrnewsymbol[description={Distributional neural network}]{DDNN}{DDNN}


%E
\glsxtrnewsymbol[description={Exploratory data analysis}]{EDA}{EDA}
\glsxtrnewsymbol[description={Electricity forecasting}]{EF}{EF}
\glsxtrnewsymbol[description={Electricity load forecasting}]{ELF}{ELF}
\glsxtrnewsymbol[description={Empirical mode decomposition}]{EMD}{EMD}
\glsxtrnewsymbol[description={Electricity load forecasting}]{EPF}{EPF}
\glsxtrnewsymbol[description={Exponential smoothing models}]{ESM}{ESM}
\glsxtrnewsymbol[description={Extract transform load}]{ETL}{ETL}
\glsxtrnewsymbol[description={Electric vehicles}]{EVs}{EVs}

%G
\glsxtrnewsymbol[description={Gradient boosting regression tree}]{GBRT}{GBRT}
\glsxtrnewsymbol[description={Global energy forecasting competition}]{GEFCom}{GEFCom}
\glsxtrnewsymbol[description={Gaussian process regression}]{GPR}{GPR}


%H
\glsxtrnewsymbol[description={Holt-Winters-Taylor exponential smoothing method}]{HWT}{HWT}

%K
\glsxtrnewsymbol[description={Kernel density estimation}]{KDE}{KDE}


%L
\glsxtrnewsymbol[description={Low carbon technologies}]{LCTs}{LCTs}
\glsxtrnewsymbol[description={Least squares support vector regression}]{LSSVR}{LSSVR}

\glsxtrnewsymbol[description={Low voltage}]{LV}{LV}

%M
\glsxtrnewsymbol[description={Mean absolute error}]{MAE}{MAE}
\glsxtrnewsymbol[description={Mean absolute scaled error}]{MASE}{MASE}
\glsxtrnewsymbol[description={Mean absolute percentage error}]{MAPE}{MAPE}
\glsxtrnewsymbol[description={Minimal gated memory network}]{MGM}{MGM}

\glsxtrnewsymbol[description={Multiple linear regression}]{MLR}{MLR}

%O
\glsxtrnewsymbol[description={Ordinary least squares}]{OLS}{OLS}

%R
\glsxtrnewsymbol[description={Random forest}]{RF}{RF}
\glsxtrnewsymbol[description={Reproducing kernel hilbert space}]{RKHS}{RKHS}
\glsxtrnewsymbol[description={Root mean squared error}]{RMSE}{RMSE}


%P
\glsxtrnewsymbol[description={Prediction interval}]{PI}{PI}
\glsxtrnewsymbol[description={Price Forecasting}]{PF}{PF}

\glsxtrnewsymbol[description={Probabilistic Price Forecasting}]{PPF}{PPF}

%Q
\glsxtrnewsymbol[description={Quantile regression}]{QR}{QR}

\glsxtrnewsymbol[description={Quantile regression averaging}]{QRA}{QRA}

%S
\glsxtrnewsymbol[description={Small and medium-sized enterprises}]{SME}{SME}

\glsxtrnewsymbol[description={Smoothed nonparametric ARX}]{SNARX}{SNARX}

\glsxtrnewsymbol[description={Sustainable development goals}]{SDGs}{SDGs}
\glsxtrnewsymbol[description={Support vector machine}]{SVM}{SVM}

%T
\glsxtrnewsymbol[description={Time of use tariffs}]{TOU}{TOU}
\glsxtrnewsymbol[description={Transmission system operator}]{TSO}{TSO}



%MATHEMATICAL NOTATION
% D
\glsxtrnewsymbol[description={dual optimum}]{d^*}{\ensuremath{d^*}}

%E
\glsxtrnewsymbol[description={expectation}]{mathbb{E}}{\ensuremath{\mathbb{E}}}


% F
\glsxtrnewsymbol[description={feature matrix}]{Phi(x)}{\ensuremath{\Phi(x)}}

\glsxtrnewsymbol[description={feature space}]{F}{\ensuremath{F}}

\glsxtrnewsymbol[description={feature vector}]{phi(x)}{\ensuremath{\phi(x)}}

% G
\glsxtrnewsymbol[description={dual function}]{g(v)}{\ensuremath{g(v)}}

% I
\glsxtrnewsymbol[description={Indicator function}]{mathbb{1}}{\ensuremath{\mathbb{1}}}


% K
\glsxtrnewsymbol[description={kernel matrix}]{K}{\ensuremath{K}}

\glsxtrnewsymbol[description={kernel function}]{k(x,x')}{\ensuremath{k(x,x')}}

% L
\glsxtrnewsymbol[description={lagrangian}]{L}{\ensuremath{L}}

% P
\glsxtrnewsymbol[description={pinball loss}]{rho_q}{\ensuremath{\rho_q}}

\glsxtrnewsymbol[description={primal optimum}]{p^*}{\ensuremath{p^*}}

% R
\glsxtrnewsymbol[description={relative interior}]{relint}{\textrm{relint}}

\glsxtrnewsymbol[description={reproducing kernel hilbert space}]{H}{\ensuremath{H}}

\printunsrtglossary[title=List of Symbols, type=symbols,style=long]


\appendix

\chapter{Appendix}
\chapter{Appendix}
\section{Feature map normalization}\label{appendix:new_feature}
\begin{proof}
\begin{align*}
    \|
    \phi^{new}(x)\|_{\mathcal{H}}^{2} &= \left\|\frac{\phi(x)}{\|\phi(x)\|_{\mathcal{H}}}
    \right\|_{\mathcal{H}}^{2}
    \\
    &=
    \left\|
    \frac{\phi(x)}
    {\sqrt{k(x,x)}}
    \right\|_{\mathcal{H}}^{2}
    \\
    &=
    \langle
    \frac{\phi(x)}
    {\sqrt{k(x,x)}}
    ,
    \frac{\phi(x)}
    {\sqrt{k(x,x)}}
    \rangle_{\mathcal{H}}
    \\
    &=
    \frac{1}{\sqrt{k(x,x)^{2}}}
    \langle
    \phi(x)
    ,
    \phi(x)
    \rangle_{\mathcal{H}}
    \\
    &=1
\end{align*}
\end{proof}



\section{Quantile regressor extensive comparison}\label{appendix:quantile_regressor_extensive_comparison}
This section contains extensive comparison between our kernel quantile regression and other state of the art quantile regressors. We benchmark it on popular machine learning datasets.
\subsection{Boston housing dataset}
The Boston housing dataset \href{https://www.kaggle.com/datasets/altavish/boston-housing-dataset}{https://www.kaggle.com/datasets/altavish/boston-housing-dataset} contains information about various attributes for suburbs in Boston.
There are 13 indipendent variables:
\begin{itemize}
\item CRIM per capita crime rate by town
\item ZN proportion of residential land zoned for lots over 25,000 sq.ft.
\item INDUS proportion of non-retail business acres per town.
\item CHAS Charles River dummy variable (1 if tract bounds river; 0 otherwise)
\item NOX nitric oxides concentration (parts per 10 million)
\item RM average number of rooms per dwelling
\item AGE proportion of owner-occupied units built prior to 1940
\item DIS weighted distances to five Boston employment centres
\item RAD index of accessibility to radial highways
\item TAX full-value property-tax rate per 10,000
\item PTRATIO pupil-teacher ratio by town
\item B $1000(Bk - 0.63)^2$ where Bk is the proportion of afroamericans by town
\item LSTAT lower status of the population
\end{itemize}
The dependent variable is MEDV, that is the median value of owner occupied homes in \$1000's

\begin{table}
\caption{Pinball loss Boston housing data}
\begin{tabular}{lllll}
\toprule
    & Linear qr & Gbm qr & Quantile forest & Kernel qr \\
\midrule
0 & 13.785678 & 11.418540 & 10.587686 & 10.297572 \\
\bottomrule
\end{tabular}
\end{table}

\begin{table}
    \caption{Pinball loss quantile-wise Boston data}
\begin{tabular}{lllll}
\toprule
    & Linear qr & Gbm qr & Quantile forest & Kernel qr \\
\midrule
0.100000 & 0.729749 & 0.771714 & 0.588441 & 0.578898 \\
0.200000 & 1.122582 & 1.033442 & 0.932824 & 0.869145 \\
0.300000 & 1.479486 & 1.170642 & 1.153765 & 1.142783 \\
0.400000 & 1.712577 & 1.436263 & 1.352667 & 1.331955 \\
0.500000 & 1.911385 & 1.344361 & 1.408333 & 1.396300 \\
0.600000 & 1.989514 & 1.448885 & 1.464902 & 1.431705 \\
0.700000 & 1.938362 & 1.508741 & 1.427912 & 1.382772 \\
0.800000 & 1.658058 & 1.497901 & 1.275059 & 1.245288 \\
0.900000 & 1.243965 & 1.206591 & 0.983784 & 0.918725 \\
\bottomrule
\end{tabular}
\end{table}
        
\begin{table}
\caption{Mean absolute error Boston data}    
\begin{tabular}{lllll}
\toprule
    & Linear qr & Gbm qr & Quantile forest & Kernel qr \\
\midrule
0 & 3.826326 & 2.845989 & 2.965490 & 2.810494 \\
\bottomrule
\end{tabular}

\end{table}

\subsection{Abalone dataset}
The abalone data \href{https://archive.ics.uci.edu/dataset/1/abalone}{https://archive.ics.uci.edu/dataset/1/abalone} consist of measurements of abalone molluscs, the goal is predicting their age by building a model for estimating its number of rings; age is the number of rings plus 1.5
The data has 8 attributes:
\begin{itemize}
    \item Sex Categorical variable either male, female or infant
    \item Length
    \item Diameter
    \item Height
    \item Whole height
    \item Shucked height
    \item Viscera weight
    \item Shell weight
\end{itemize}

\begin{table}
    \caption{Pinball loss Abalone data}
\begin{tabular}{lllll}
    \toprule
     & Linear qr & Gbm qr & Quantile forest & Kernel qr \\
    \midrule
    0 & 5.613975 & 5.531938 & 5.212990 & 5.252491 \\
    \bottomrule
    \end{tabular}
\end{table}
    
\begin{table}
    \caption{Pinball loss quantile-wise Abalone data}
    \begin{tabular}{lllll}
    \toprule
     & Linear qr & Gbm qr & Quantile forest & Kernel qr \\
    \midrule
    0.100000 & 0.277903 & 0.290531 & 0.274079 & 0.269287 \\
    0.200000 & 0.469361 & 0.488079 & 0.453110 & 0.457286 \\
    0.300000 & 0.621961 & 0.625633 & 0.580766 & 0.596791 \\
    0.400000 & 0.729757 & 0.715875 & 0.689904 & 0.691310 \\
    0.500000 & 0.794695 & 0.766185 & 0.735945 & 0.740834 \\
    0.600000 & 0.810691 & 0.785769 & 0.744928 & 0.746636 \\
    0.700000 & 0.769587 & 0.730318 & 0.700287 & 0.710392 \\
    0.800000 & 0.667776 & 0.656913 & 0.609378 & 0.608334 \\
    0.900000 & 0.472244 & 0.472635 & 0.424593 & 0.431621 \\
    \bottomrule
    \end{tabular}
\end{table}
    
\begin{table}
    \caption{Mean absolute error Abalone data}
    \begin{tabular}{lllll}
    \toprule
     & Linear qr & Gbm qr & Quantile forest & Kernel qr \\
    \midrule
    0 & 1.627627 & 1.574179 & 1.499522 & 1.498583 \\
    \bottomrule
    \end{tabular}
\end{table} 

\subsection{Vehicle dataset}
This data contains info about used cars \href{https://www.kaggle.com/datasets/nehalbirla/vehicle-dataset-from-cardekho}{https://www.kaggle.com/datasets/nehalbirla/vehicle-dataset-from-cardekho}, the predictors are:
\begin{itemize}
    \item Year
    \item Present\_price ex showroom price
    \item Kms Driven
    \item Fuel type
    \item Seller type
    \item Transmission
    \item Owner number of previous owners
\end{itemize}
The dependent variable is the selling price.

\begin{table}
\caption{Pinball loss Vehicle data}
\begin{tabular}{lllll}
    \toprule
     & Linear qr & Gbm qr & Quantile forest & Kernel qr \\
    \midrule
    0 & 4.054449 & 2.289554 & 2.844410 & 2.204343 \\
    \bottomrule
    \end{tabular}
\end{table}

\begin{table}
    \caption{Pinball loss quantile-wise Vehicle data}
    \begin{tabular}{lllll}
    \toprule
     & Linear qr & Gbm qr & Quantile forest & Kernel qr \\
    \midrule
    0.100000 & 0.254649 & 0.139849 & 0.170489 & 0.182490 \\
    0.200000 & 0.403772 & 0.236339 & 0.285875 & 0.223165 \\
    0.300000 & 0.548820 & 0.244086 & 0.357375 & 0.242963 \\
    0.400000 & 0.576918 & 0.263169 & 0.389305 & 0.262835 \\
    0.500000 & 0.554367 & 0.306123 & 0.410738 & 0.295878 \\
    0.600000 & 0.563046 & 0.335363 & 0.398125 & 0.306062 \\
    0.700000 & 0.516019 & 0.287490 & 0.326572 & 0.283716 \\
    0.800000 & 0.407742 & 0.261882 & 0.313698 & 0.235793 \\
    0.900000 & 0.229115 & 0.215253 & 0.192233 & 0.171440 \\
    \bottomrule
    \end{tabular}
\end{table}

\begin{table}
    \caption{Mean absolute error Vehicle data}
    \begin{tabular}{lllll}
    \toprule
     & Linear qr & Gbm qr & Quantile forest & Kernel qr \\
    \midrule
    0 & 1.117714 & 0.606971 & 0.752197 & 0.594292 \\
    \bottomrule
    \end{tabular}
\end{table}
What can be concluded from these numerical examples is that, on average kernel quantile regression yields better results than quantile forest \cite{meinshausen2006quantile} and gradient boosting machine quantile regression \cite {friedman2001greedy} in terms of the pinball loss as well as in terms of the mean absolute error.



\section{Cross validation}\label{appendix:cross_validation}
Cross validation directly estimates the expected test error
\begin{equation}
    Err=\mathbb{E}\left[L\left(Y,\hat{f}(X)\right)\right]=\mathbb{E}\left[Err_{\Tau}\right]
\end{equation}
$Err_{\Tau}$ is the prediction error over an independent test sample
\begin{equation}
    \mathbb{E}_{\Tau}=\mathbb{E}\left[L\left(Y,\hat{f}(X)\right)| \Tau \right]
\end{equation}
That is $Err$ is the average over everything that is random: X,Y and the training set $\Tau$ used to learn $\hat{f}$. Hence, our interest lies in estimating the $Err$ quantity in order to guide model selection.
\subsection{K-fold cross validation}
K-fold cross validation splits the data into K roughly equally sized parts, then K models are trained. For k from 1 to K we train the kth model on the whole dataset except the kth partition. Next,  the prediction error of the kth fitted model is computed. Finally averaging all the k prediction erros we obtain an estimate for the expected test error. We select the model which performs best in terms of expected prediction error.
\\
Usually, K is set equal to 5 or 10. Leave-one-out cross validation is the case when K is set equal to the size of the data.
% REMEMBER: Parameters allow the model to learn the rules from the data while hyperparameters control how the model is training. Parameters learn their own values from data. In contrast, hyperparameters do not learn their values from data. We need to manually specify them before training the model
\subsection{Gridsearch}
Model performance depends higly on the choice of hyperparameters. Notice, there is no way to get to know them in advance, therefore, all we can do is trying a lot of combinations until we fit a good enough set of hyperparameters. Essentially, gridsearch carries out hyperparameter tuning by performing a search over a predefined hyperparemeters grid. During its search, it tries all possible combination and evaluates the different models %(different hyperparameters define different models)
using cross validation. For example, suppose that the grid contains 100 possible candidates and that we are doing 5-fold cross-validation, then the gridsearch algorithm will carry out 500 iterations.
Therefore, we get an estimate of prediction error for each considered model and this guides us in hyperparameters (model) selection.

\subsection{Randomized search}
Randomized search controls the number of steps by choosing smartly the hyperparameters to try in each iterations.
Let us say, there are 100 candidates and we set the number of iterations to 20 then the search will stop after the 20th iteration and return the best set among the hyperparameters observed.
The advantage is that it is much more quicker than gridsearch, but on the other hand its performance is worse than gridsearch.

\subsection{Halving gridsearch}
Gridsearch has been the to go choice for hyperparameters tuning for the past years. However, such method is brute forcing all possible combinations, hence it is highly computationally intensive, especially when it comes to large datasets.
\\
The scikit-learn team addressed this disadvantage of gridsearch by introducing the halving gridsearch method \cite{scikithalvinggridsearch} (2020). Such technique has proved itself to greatly speed up hyperparameter tuning.
The ground concept underlying this method is successive halving. During the first iteration of halving gridsearch, all candidates are trained on a small subset of the training set. Next, we keep only the candidates which performed best and compare  them again on a bigger subset of the training set. As the iterations pass, the surviving candidates will be given more and more training samples. The algorithm stops when we are left with only the best set of hyperparameters.

\subsection{Cross validation for time series data}
When data points are dependent on preceding values, we cannot use standard K-fold cross validation. The rationale is that K-fold will randomize the order of the data, thus it might happen to use future data to predict the past; we want to avoid such behaviour in a time series setting. 
When carryout out any kind of cross validation, we must keep consistency in the way we evaluate our predictors during model selection and in the way we perform evaluation of the test data.
Hence, for time series crossvalidation we have the timeseries split procedure, see figure \ref{fig:crossvalidationtimeseries} for a visualisation; the blue observations make up the training sets while the orange observations form the test sets.
\begin{figure}
    \includegraphics[width=\textwidth]{images/crossvalidationtimeseries.png}
    \caption{Cross validation for one step ahead timeseries data \cite{hyndman2018forecasting}}
    \label{fig:crossvalidationtimeseries}
\end{figure}
We then average all the observed losses in order to get an estimate for $Err$.
Notice, in the literature this procedure is sometimes also referred to evaluation on a rolling forecasting origin. This comes from the fact that at each iteration we push forward the origin of our forecast.
The same concept applies for multi step ahead forecasting, see picture \ref{fig:crossvalidationtimeseries2}.
That is in predicting $\hat{L}_{N+m}$ we use as inputs $L_1, L_2, \dots, \hat{L}_{N},\hat{L}_{N+1},\dots, \hat{L}_{N+m-1}$; where $\hat{L}_{N},\hat{L}_{N+1},\dots, \hat{L}_{N+m-1}$ are one step ahead forecasts.
\begin{figure}
    \includegraphics[width=\textwidth]{images/crossvalidationtimeseries2.png}
    \caption{Cross validation for m step ahead timeseries data \cite{hyndman2018forecasting}}
    \label{fig:crossvalidationtimeseries2}
\end{figure}

\section{Kernel methods best practices}
Following, are reported a couple of considerations important to keep in mind when working with kernel methods.

\subsection{Data normalization}\label{appendix:normalization}
With data normalization we transform the range of the feautures to a standard scale.
Such preprocessing step is essential when employing distance based algorithms like svm or k-nearest neighbors. The rationale behind it is that by normalizing data we give an uniform weight to each feature in the learning process; in this way we do not favour larger scale features.
Examples of the most popular features scaling are:
\begin{itemize}
    \item Standard scaler= it computes the standard score z of a sample x
    \\
    z=$\frac{x-\mu}{\sigma}$.
    \item MinMax scaler= it maps every data sample to the range 0, 1.
    \\
    $\frac{x-\min(X)}{\max(X)-\min(X)}$
    \item Robust scaler= it scales features using statistics that are robust to outliers.
    Essentially, it subtracts the median and then scales the data according to the interquantile range.
\end{itemize}
Many other data scaling algorithms exists, we refer the reader to a thorough comparison \cite{scikitscalers}.
\\
We conclude this subsection with a custom example that motivates the need of feature scaling \cite{scikitscale_example}. The idea is to compare the results of modelling the data with k nearest neighbors on the unscaled data against the scaled data. The considered data is the wine recognition dataset \href{https://archive.ics.uci.edu/dataset/109/wine}{https://archive.ics.uci.edu/dataset/109/wine}. The goal for this dataset is recognising from whose cultivator the wine comes based on two features with a completely different scale. The first feature has values in the [0,1000] range while the second feature has values contained in [1,10].
The unscaled and scaled version are compared in figure \ref{fig:feature_scaler_example}
\begin{figure}[!h]
    \includegraphics[width=\textwidth]{images/feature_scaler_example.png}
    \caption{importance of feature scaling}
    \label{fig:feature_scaler_example}
\end{figure}
What we can conclude from the image is that, the model trained on scaled data is greatly better than the other.
On the left, we can see that distances between categories are impacted solely by the larger scale feature. Conversely, on the right, we have that the two features contributing equally in determining the neighbors.

\subsection{Data compression}
When the number of points n is large, kernel methods suffer from high computational costs. 
Using kernel methods, we have that the storage cost is of the order of $O(n^2)$ while the computational cost for finding the solution is of the order $O(n^3)$.
A significant speed up can be obtained thanks to low rank approximation.


\subsubsection{Nystrom decomposition}
The Nyström approximation involves storing a submatrix of the whole kernel matrix. Thus, storage and computational cost are reduced to $O(nm)$ and $O(nm^2)$ respectively.
Nyström works by selecting $m<n$ points, callled representative points; it approximates K as
\begin{equation}
    \tilde {K}=K_{n,m} K_{m,m}^{-1}K_{n,m}^\intercal
\end{equation}


\subsubsection{Pivoted Cholesky decomposition}
Pivoted Cholesky approximate the Cholesky decomposition of a matrix. Since kernel matrices are positive definite, they can decomposed in terms of the Cholesky decomposition. Hence, we have that pivoted Cholesky can be used to approximate the full kernel matrix.


\newpage
\section{Source code}\label{src_code}
The whole code for the project is hosted on
\url{https://github.com/luca-pernigo/ThesisKernelMethods}\label{github_repo}.
\\
\begin{itemize}
    \item query: folder containing Scopus data and scripts to generate bibliometric survey plots in section \ref{literature_review}
    \item kqr.py: file implementing our custom kernel quantile regression
\end{itemize}


\backmatter

\bibliographystyle{plain}
\bibliography{refs}

% %Include all references
\nocite{*}

% sitography
\begingroup
\raggedright
\sloppy
\bibliographystyleW{plain}
\bibliographyW{site}
\endgroup

% include pdf file, first we have to flatten it to show modifications
\includepdf[pages=-]{declaration-originality-flat.pdf}
\end{document}
