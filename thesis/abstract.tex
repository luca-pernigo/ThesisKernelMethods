

\begin{abstract}
  % introduction
  This thesis is concerned with electricity forecasting. 
  Specifically, we concentrated on the probabilistic framework. This choice was motivated by its importance in systems planning and operations as a consequence of the inception of competitive power markets, smart grids and renewable integration requirements.
  % methods
  In doing so, we focused on the family of kernel methods. In addition, we also compared them against several statistical and machine learning techniques.
  % results
  Our results showed the goodness of kernel methods in the field of electricity forecasting both in terms of point and probabilistic forecasting. In particular, in the probabilistic context, our experiments showed the validity of kernel quantile regression equipped with the absolute Laplacian kernel.
  % discussion
  These findings indicate kernel methods are well suited to the characteristic of electricity.
  Anyone interested in forecasting energy should consider them when faced with the choice of the model.
  They can be employed stand alone or combined with other valid methods into ensembles.
  

  %This example thesis briefly shows the main features of our thesis
  %style, and how to use it for your purposes.
\end{abstract}
  % The theory of kernel methods will be applied to the problem of point and probabilistic forecasting for the energy sector.
  %   Such choice is motivated by its interesting implications.